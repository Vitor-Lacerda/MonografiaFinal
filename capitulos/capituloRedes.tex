\chapter{Redes Neurais Artificiais} \label{cap:redes}

Comparado com os computadores mais avançados que existem hoje, o cérebro humano ainda se mostra muito mais poderoso e eficaz do que as máquinas. O cérebro é capaz de processar informação a uma velocidade muito maior do qualquer computador convencional e realiza com facilidade tarefas como o reconhecimento de padrões e classificação de objetos, enquanto algoritmos tradicionais falham. 

Redes neurais artificiais foram criadas com inspiração no funcionamento e na estrutura do cérebro, como uma alternativa poderosa para resolver problemas que as arquiteturas tradicionais não eram capazes de resolver eficientemente. Essas redes emulam a estrutura cerebral natural, se utilizando de elementos distintos de processamento (neurônios) que se comunicam para realizar a tarefa desejada. 

Simon Haykin\cite{Haykin} define uma rede neural como "...um processador distribuído massivamente paralelo composto por unidades simples de processamento, que possui uma propensidade natural a armazenar conhecimento experimental e torná-lo disponível para uso."

As redes neurais artificiais são utilizadas para resolver problemas como ajuste de funções, classificação de objetos e reconhecimento de padrões. Nesse capítulo, serão explicados conceitos importantes para o entendimento das redes artificiais, começando pelo seu elemento mais básico, o neurônio.

\section{Neurônios naturais}


\section{Neurônios artificiais}


\section{Classificação de padrões}


\section{Redes Neurais Feed-Forward}