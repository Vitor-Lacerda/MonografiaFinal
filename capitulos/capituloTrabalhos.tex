\chapter{Trabalhos Correlatos} \label{cap:trabalhos}

Nesse capítulo serão expostas soluções para o problema apresentado no capítulo \ref{cap:intro}, acompanhadas de uma discussão sobre vantagens e desvantagens, e as razões pela escolha da solução proposta no capítulo \ref{cap:solucao}.

Podemos separar os métodos de detecção de vagas em duas grandes categorias: métodos intrusivos e não-intrusivos. Métodos intrusivos são caracterizados por exigirem instalações mais complexas, envolvendo a inserção de equipamento no asfalto do estacionamento ou em um estrutura de concreto. Os métodos não intrusivos normalmente se utilizam de equipamentos externos, que não exigem obras para a sua instalação.

Um exemplo comum de método itrusivo é o uso de sensores individuais em cada vaga do estacionamento. Diversos tipos de sensores podem ser utilizados, cada um com vantagens e desvantagens próprias \cite{idris09}. Sensores infravermelhos versáteis e de simples instalação, mas podem ser afetados por condições ambientais. Tubos pneumáticos e sensores de peso sob o asfalto têm alta cofiabilidade, mas exigem grandes obras para a instalação. 

Alguns tipos de sensores podem ser instalados de forma não intrusiva. Esses sensores podem ser afixados no teto ou em uma estrutura próxima a vaga. Sensores ultrassônicos por exemplo, pertencem a essa categoria. Eles transmitem ondas sonoras de baixa frequência e usam a energia refletida para determinar a ocupação das vagas \cite{kianpisheh2012smart}.

Intrusivos ou não, os métodos de detecção que utilizam sensores compartilham duas desvantagens: é necessário a instalação do sensor em cada uma das vagas, aumentando o custo do sistema e diminuindo sua escalabilidade, e eles não são adequados para estacionamentos descobertos, onde é impossível instalar sensores no teto e obras no asfalto rodoviário podem danificá-lo permanentemente \cite{idris09}.

Uma solução que se mostrou adequada para os estacionamentos descobertos foi o uso de \textit{softwares} de visão computacional, que analisam imagens de câmeras de vídeo para determinar a ocupação das vagas do estacionamento. Essa opção tem um custo baixo e exige apenas que sejam instaladas câmeras em pontos estratégicos do estacionamento. Uma vez que as câmeras foram instaladas e as imagens estão sendo capturadas, resta que seja executado um programa que as analise.

Para a escolha da abordagem para este trabalhos foram escolhidos os seguintes critérios principais:

\begin{itemize}
	
	\item O sistema deve poder ser instalado e começar sua execução em um estacionamento em qualquer estado, sem a necessidade de iniciar com todas as vagas descocupadas, possibilitando uma instalação a qualquer momento, sem interrupção das operações do estacionamento.
	\item O sistema deve necessitar de interação humana mínima, se limitando apenas uma rápida calibração no início de sua execução a fim de evitar erro humano.
	\item O sistema deve ser capaz de identificar a posição das vagas do estacionamento após um certo tempo de execução, e não através de entrada manual, para compensar por erros na calibração.
	\item O sistema deve, além de determinar o número de vagas livres no estacionamento, ser capaz de indicar a posição aproximada de tais vagas de forma a facilitar ainda mais a busca por uma vaga desocupada.
	
\end{itemize}

Em \cite{delibaltov2013parking}, Delibaltov \textit{et al} descrevem um método de detecção que estima um volume tridimensional para cada vaga marcada na imagem e depois se utiliza de redes neurais para determinar \textit{pixels} da imagem pertencentes a veículos. Em seguida a sistema computa a probabilidade de que uma certa vaga esteja ocupada baseado em uma função de probabilidade e os \textit{pixels} de veículo na região de interesse de cada vaga. Esse método se mostra preciso e eficaz, mas exige que o estacionamento esteja vazio e que cada vaga seja marcada individualmente no momento de execução, de forma que ele não é capaz de mapear as vagas do estacionamento ou detectar veículos estacionados irregularmente.

Bong \cite{bong2008integrated} propõe um sistema que se utiliza de subtração de imagens e o uso de detecção automática de coordenadas aproximadas das vagas para resolver este problema. Porém esse sistema exige que seja armazenada uma imagem do estacionamento completamente vazio, além de que seja criada uma imagem que marca cada vaga para o processo de inicialização do sistema.

Um sistema que se utiliza de uma técnica de classificação de histogramas de crominância em áreas de interesse e um algoritmo de que encontra pontos com \textit{features} relevantes foi proposto em \cite{true2007vacant}. Esse sistema tem a vantagem de ser relativamente invariante quanto a iluminação da imagem, por utilizar os canais de crominância para a análise. Também é bastante capaz de resolver o problema de oclusão de veículos atrás de outros na imagem de câmeras com uma angulação maior.

A solução proposta neste trabalho procura compensar alguns dos problemas presentes nos trabalhos mencionados, enquanto apresenta uma nova abordagem a ser considerada e aprimorada por trabalhos futuros.

