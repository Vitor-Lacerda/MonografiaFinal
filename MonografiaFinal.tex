%%%%%%%%%%%%%%%%%%%%%%%%%%%%%%%%%%%%%%%%
% Classe do documento %
%%%%%%%%%%%%%%%%%%%%%%%%%%%%%%%%%%%%%%%%

% Nós usamos a classe "unb-cic".  Deixe apenas uma das linhas
% abaixo não-comentada, dependendo se você for do bacharelado ou
% da licenciatura.

\documentclass[bacharelado]{unb-cic}
%\documentclass[licenciatura]{unb-cic}



%%%%%%%%%%%%%%%%%%%%%%%%%%%%%%%%%%%%%%%%
% Pacotes importados
%%%%%%%%%%%%%%%%%%%%%%%%%%%%%%%%%%%%%%%%

\usepackage[brazil,american]{babel}
\usepackage[T1]{fontenc}
\usepackage{indentfirst}
\usepackage{natbib}
\usepackage{amsmath}
\usepackage{mathtools}
\usepackage{xcolor,graphicx,url}
\usepackage{subcaption}
\usepackage{url}
\usepackage[utf8]{inputenc}


%%%%%%%%%%%%%%%%%%%%%%%%%%%%%%%%%%%%%%%%
% Caminho das imagens
%%%%%%%%%%%%%%%%%%%%%%%%%%%%%%%%%%%%%%%%%
\graphicspath{ {imagens/} }


%%%%%%%%%%%%%%%%%%%%%%%%%%%%%%%%%%%%%%%%
% Cores dos links
%%%%%%%%%%%%%%%%%%%%%%%%%%%%%%%%%%%%%%%%

% Veja o arquivos cores.tex se quiser ver que outras cores estão
% pré-definidas.  Utilizando o comando \hypersetup abaixo nós
% evitamos aquelas caixas vermelhas feias em volta dos links.

\input{cores}
\hypersetup{
  colorlinks=true,
  linkcolor=DarkScarletRed,
  citecolor=DarkScarletRed,
  filecolor=DarkScarletRed,
  urlcolor= DarkScarletRed
}



%%%%%%%%%%%%%%%%%%%%%%%%%%%%%%%%%%%%%%%%
% Informações sobre a monografia
%%%%%%%%%%%%%%%%%%%%%%%%%%%%%%%%%%%%%%%%

\title{Detecção e monitoramento de vagas disponíveis em estacionamentos abertos através de processamento de imagens}

\orientador{\prof \dr Alexandre Zaghetto}{CIC/UnB}
%\coorientador[a]{\prof[a] \dr[a] Coorientadora}{MAT/UnB}
\coordenador{\prof \dr Rodrigo Bonifácio de Almeida}{CIC/UnB}
\diamesano{28}{Novembro}{2016}

\membrobanca{\prof \dr Flávio Barros Vidal}{CIC/UnB}
\membrobanca{\prof \dr Luiz Henrique}{CIC/UnB}

\autor{Vitor de Alencastro}{Lacerda}
\CDU{004.4}

\palavraschave{Estacionamento, Vagas livres, Processamento de Imagens, Redes Neurais, Fluxo Óptico, GLCM }
\keywords{Parking lots,Free Spaces, Image processing, Neural Networks, Optical Flow, GLCM}



%%%%%%%%%%%%%%%%%%%%%%%%%%%%%%%%%%%%%%%%
% Texto
%%%%%%%%%%%%%%%%%%%%%%%%%%%%%%%%%%%%%%%%


\begin{document}
  \maketitle
  \pretextual

  \begin{dedicatoria}
  Dedico a todos os outros motoristas frustrados que gostariam de perder menos tempo estacionando seus carros.
  \end{dedicatoria}

  \begin{agradecimentos}
  Agradeço a minha família e amigos pelo apoio durante toda a longa duração do curso e da elaboração deste trabalho e ao professor Alexandre Zaghetto pela orientação.
  \end{agradecimentos}

  \begin{resumo}
  Encontrar vagas livres em grandes estacionamentos muitas vezes é uma tarefa frustrante e que consome muito tempo. Soluções que ajudam um motorista a encontrar uma vaga mais rapidamente economizam tempo e dinheiro, além de servir como fator diferencial entre estabelecimentos comerciais. Este trabalho apresenta um programa que analisa a imagem de uma câmera que filma um estacionamento e determina a ocupação das vagas presentes na imagem através da detecção de movimento no vídeo e uma rede neural artificial. Uma solução como esta, que utiliza técnicas de processamento de imagens, tem custo de instalação e manutenção muito menor do que o uso de sensores individuais nas vagas, além de ser muito mais adequada para uso em grandes estacionamentos descobertos.
  \end{resumo}

  \selectlanguage{american}
  \begin{abstract}
  Finding free parking spaces in big parking lots is often a frustrating and time-consuming task. Solutions that help drivers find spaces faster save both time and money, and can help attract customers to commercial establishments. This work presents an approach that analyzes images from a camera filming a parking lot and determines parking space occupancy through movement detection and an artificial neural network. Solutions that use image processing techniques have much lower installation and maintenance costs than solutions that involve individual sensors. They are also more adequate for outdoor parking lots.
  \end{abstract}
  \selectlanguage{brazil}

  \tableofcontents
  \listoffigures
  \listoftables

  \textual
  \chapter{Introdução}\label{cap:intro}

			Atualmente, a frota de carros dos grandes centros urbanos tem fica cada vez maior. Esse aumento no número de veículos cria uma demanda por espaço de estacionamento, levando a construção de novos estacionamentos e uma dificuldade muito maior de se encontrar vagas disponíveis nos estacionamentos existentes nos estabelecimentos comerciais, áreas residenciais e centros urbanos.
			
			De fato, todo ano, milhões de pessoas gastam milhares de horas rondando estacionamentos de supermercados, \textit{shoppings} e prédios comerciais em busca de uma vaga de estacionamento vazia. Procurar por uma vaga em um estacionamento grande e cheio é uma tarefa cansativa e desagradável. Muitas vezes vagas desocupadas não são encontradas por motoristas que preferem parar seu carro e esperar que uma vaga seja liberada próximo a ele. Outras vezes possíveis clientes desistem de ir a um determinado estabelecimento pois sabem da dificuldade de estacionar que enfrentarão. Esse problema é comum principalmente em \textit{shopping centers} e grandes supermercados, que possuem seus próprios estacionamentos disponíveis para os clientes, mas que estão frequentemente muito cheios.
			
			Facilitar a tarefa da busca de vagas em estacionamentos de estabelecimentos como estes é benéfico então não só para os clientes, mas também para os donos e gerentes desses estabelecimentos. Sendo assim, esse trabalho tem como objetivo criar um sistema que seja capaz de determinar a ocupação das vagas de um estacionamento descoberto e informar quantas vagas estão disponíveis e a sua localização aproximada para os motoristas através do processamento de imagens do estacionamento. Tal solução facilitaria a procura de vagas, diminuindo o tempo gasto trafegando por fileiras de vagas ocupadas.
			
			A solução apresentada utiliza redes neurais artificais combinadas com técnicas descritoras de textura e características de crominância para diferenciar imagens de carros de imagens de vagas desocupadas. Cada região que contém vagas na imagem obtida é dividida em diversas seções verticais. Cada seção será classificada entre vaga vazia ou veículo, de forma que o programa então é capaz de determinar quantas vagas estão ocupadas nessa região e, consequentemente, quantas estão vazias. Algumas soluções propostas em trabalhos correlatos serão apresentadas no capítulo \ref{cap:trabalhos}.
			
			No capítulo \ref{cap:fundament} serão apresentados alguns conceitos importantes para o entendimento da solução que será apresentada posteriormente. Serão detalhados o que é uma imagem, como funciona um vídeo e outros elementos teóricos importantes como espaços de cores e descritores de textura. O capítulo \ref{cap:redes} apresenta uma fundamentação teórica de redes neurais artificiais. Esse assunto foi separado em um capítulo próprio por causa da grande quantidade de conteúdo e por sua importância especial neste trabalho.
			
			No capítulo \ref{cap:solucao}, a solução proposta é apresentada. Nesta seção será detalhado o funcionamento do sistema criado, o fluxo de execução do programa e os elementos que permitem o funcionamento do sistema.
			
			Em seguida, o capítulo \ref{cap:results} apresentará resultados obtidos em testes que comparavam o sistema a observadores humanos, seguidos de observações sobre os resultados do programa. Esse capítulo também apresenta os detalhes da rede neural artificial utilizada.
			
			O capítulo \ref{cap:conclusao} possui comentários sobre a taxa de sucesso do trabalho, suas fragilidades, sua adequação para uso no estado atual e se o objetivo desejado foi alcançado. O capítulo finaliza o trabalho com um comentário sobre possíveis trabalhos futuros para a evolução do sistema.
			
			









	\chapter{Fundamentação Teórica}\label{cap:fundament}



Esse capítulo tem como objetivo apresentar conceitos que facilitarão o entendimento do conteúdo do capítulo \ref{cap:solucao}, detalhando técnicas e conceitos utilizados na execução do trabalho através de exemplos e imagens. Serão apresentados conceitos relacionados ao processamento de imagens estáticas, seguidos de uma discussão breve sobre técnias aplicadas em vídeos. Por fim serão detalhados as redes neurais e o conceito de classificação de padrões.

\section{Processamento de Imagens}\label{sec:processamento}

\subsection{Imagens em nível de cinza}

Para que um computador seja capaz de operar sobre uma imagem, é preciso que seja utilizado um modelo de representação que traduza o que os nossos sentidos conseguem perceber em informações que podem ser interpretadas por uma máquina que não é dotada de visão. A forma mais simples de se representar uma imagem são as imagens em nível de cinza. Podemos definir imagens como uma função \textit{f{x,y}} onde x e y são coordenadas espaciais e o valor de  \textit{f{x,y}} é a luminosidade, ou nível de cinza, da imagem naquele ponto. Quando esses valores são todos discretos, chamamos essa imagem de uma imagem digital.\cite{gonzalez2009digital} Cada elemento individual dessa imagem, cada valor em cada coordenada, pode ser chamado de um \textit{picture element} ou mais comumente \textit{pixel}. Em imagens digitais em nível de cinza, cada \textit{pixel} possui 1 \textit{bit} de informação, ou seja, pode assumir valores entre 0 e 255, onde o valor 0 representa o preto e 255 representa o branco. 

Uma imagem então, pode ser interpretada por um programa de computador como uma matriz $MxN$ de elementos de 1 \textit{bit}, onde M é a largura da imagem e N é a sua altura. Cada elemento $p_{i,j}$ da matriz possui a informação de luminosidade do pixel correspondente e o computador é capaz de exibir e interpretar esses valores apropriadamente. A figura \ref{fig:NivelCinza} exemplifica esse processo.

\begin{figure}
 \centering
\begin{subfigure}{.5\textwidth}
  \centering
  \includegraphics[width=.5\linewidth]{MatrizNivelCinza}
  \caption{}
  \label{exemplo:sfig1}
\end{subfigure}%
\begin{subfigure}{.5\textwidth}
  \centering
  \includegraphics[width=.5\linewidth]{ExemploNivelCinza}
  \caption{}
  \label{exemplo:sfig2}
\end{subfigure}
\caption{(a) Uma matriz com níveis de cinza e (b) a imagem correspondente}
\label{fig:NivelCinza}
\end{figure}


\subsection{Espaços de cores}

Para que um sistema de processamento de imagens possa interpretar e processar cores, é preciso criar modelos apropriados para representá-las. Chamamos os modelos de representação das cores em uma imagem um espaços de cor. Existem diversos espaços de cor, cada um importante para a realisação de tarefas e necessidades diferentes. Para a compreensão deste trabalho, é necessário, porém, entender apenas dois: o espaço RGB e o espaço YCbCr.

\subsection{O espaço RGB}


RGB é um acrônimo para \textit{Red, Green, Blue}, vermelho, verde e azul em inglês. Pesquisas mostram que um grande gama de cores pode ser formado através de combinações aditivas das cores vermelho, verde e azul. Essas cores são consideradas então cores primárias aditivas.\cite{IBGE2000introducao}. Nesse modelo, a cor de um \textit{pixel} de uma imagem é representada através de um três coeficientes que definem a influência de cada cor primária na combinação. Uma imagem no modelo RGB é então representada por três matrizes de níveis de cinza de dimensões iguais, chamadas canais, onde cada valor representado na matriz, representa o valor do coeficiente da cor correspondente na imagem final. Isto é, a imagem pode ser representada por uma matriz $MxNx3$ onde a cor final $C_{i,j}$ de cada \textit{pixel} da imagem é definida pela equação \ref{eq:corRGB}. A imagem \ref{fig:Espacos:sub:RGB} mostra os canais de uma imagem RGB separadamente.

Esse espaço de cor é o mais comumente encontrado no cotidiano, uma vez que é semelhante visão humana e portanto é utilizado pelos dispositivos multimídia mais comuns.
	
	\begin{equation}\label{eq:corRGB}
			C_{i,j} = p_{i,j,1} . R + p_{i,j,2} . G + p_{i,j,3} . B ,  (0<i<M, 0<j<N)
	\end{equation}

\subsection{Espaço YCbCr}

Assim como o espaço de cor RGB, o espaço YCbCr também representa uma imagem através de três matrizes, porém, seus canais contém informações diferentes do modelo RGB. Esse modelo, é muito utilizado para o armazenamento de vídeos, uma vez que o modelo tira vantagem de alguns aspectos da visão humana para poder armazenar menos dados, sem perda significativa de informação visual. Por exemplo, humanos são mais sensíveis à detalhes e variações em níveis de cinza do que detalhes em imagens coloridas. Além disso, o olho humano é mais sensível ao verde do que qualquer outra cor\cite{colorSpacesDigitalVideo}. Com esse conhecimento, o espaço YCbCr representa as cores de uma imagem através de sua luminosidade(Y) e os valores de crominância azul(Cb) e crominância vermelha(Cr). Cb e Cr são sinais de diferença de cor e são definidos pela subtração do valor de luminosidade do canal azul e vermelho da imagem, respectivamente. A luminosidade é definida pela equação \ref{eq:Y}\cite{LivroVideoDigital} que reflete a sensibilidade maior a cor verde da visão humana.

Na figura \ref{fig:Espacos:sub:YCbCr} estão exemplificados os canais de uma imagem no espaço YCbCr.

\begin{equation}\label{eq:Y}
	Y = 0,299.R + 0,587.G + 0,114.B

\end{equation}


\begin{figure}
 \centering
\begin{subfigure}{.5\textwidth}
  \centering
  \includegraphics[width=.8\linewidth]{exemploRGBFinal}
	\caption{}
	\label{fig:Espacos:sub:RGB}
\end{subfigure}\
\begin{subfigure}{.5\textwidth}
  \centering
  \includegraphics[width=.8\linewidth]{exemploYCbCrFinal}
	\caption{}
	\label{fig:Espacos:sub:YCbCr}
\end{subfigure}
\caption{(a) Uma imagem e cada um de seus canais RGB ordenados da esquerda para a direita e (b) os canais YCbCr da imagem ordenados da esquerda para a direita.}
\label{fig:Espacos}
\end{figure}

\subsection{Descritores de textura}\label{sec:descritores}

Descritores de textura são algoritmos que procuram fazer o que o olho humano faz com facilidade: distinguir entre tipos diferentes de objetos apenas por algumas de suas características visuais. Estes algoritmos são utilizados quando a abordagem de processamento \textit{pixel-a-pixel} se mostra insuficiente. Eles observam e analisam características pertinentes a imagem inteira e são capazes de identificar diferenças mais sutis entre imagens diferentes. Existem diversas técnicas de descrição de textura que são utilizadas com sucesso em aplicações de processamento de imagem. Para esse trabalho, a técnica escolhida foi a conhecida como GLCM(\textit{gray-level co-ocurrence matrix}) descrita na seção \ref{sec:GLCM}, por ter execução rápida e ser invariante quanto a escala de cinza.

\subsection{GLCM}\label{sec:GLCM}

GLCM, ou matriz de co-ocorrência de nível de cinza, é uma técnica de descrição de textura que visa extrair medidas estatíticas da imagem sendo analisada. Para tanto, é criada uma matriz quadrada de tamanho $MxM$ onde M é a quantidade de níveis de cinza possíveis na imagem em questão. Essa matriz armazena a probabilidade de dois \textit{pixels} se relacionarem através de uma certa relação espacial\cite{GLCM}. 

Para esse trabalho, a relação observada é a da vizinhança entre dois \textit{pixels}, sempre analisando o valor a direita de um determinado \textit{pixel} da imagem. Os 255 valores de intensidade possíveis são divididos em 8 níveis. Cada elemento $P_{i,j}$ na GLCM $G$ conta a quantidade de vezes que um \textit{pixel} com valor de intensidade no nível $j$ apareceu à direita de um \textit{pixel} com valor no nível $i$. A figura \ref{fig:GLCM} exemplifica a criação da GLCM a partir de uma imagem com 8 níveis possíveis.

\begin{figure}
\includegraphics[width=8cm]{GLCM} 
\centering
\caption{Exemplo da elaboração da GLCM. Extraída de https://www.mathworks.com/help/images/ref/graycomatrix.html}
\label{fig:GLCM}
\end{figure}

Uma vez criada a imagem, diversas medidas podem ser tiradas. Para esse trabalho são extraídas 4 características definidas abaixo. Nas equações apresentadas $P_{(i,j)}$ representa o valor do elemento na posição $(i,j)$ da GLCM, $\mu_i$ e $\mu_j$ representam a média dos valores de $i$ e $j$ respectivamente e $\sigma_i$ e $\sigma_j$ os desvios padrão destes valores.

\begin{itemize}

\item Contraste: mede o contraste entre um \textit{pixel} e seu vizinho na imagem. Deve ser 0 para uma imagem completamente homogênea. Definido por:
	\begin{equation}
		C = \sum_{i,j} (i-j)^{2}P_{(i,j)}
	\end{equation}
	
\item Correlação: mede a taxa de correlação de um \textit{pixel} e seu vizinho na imagem inteira. Definida por:
	\begin{equation}
		Co = \sum_{i,j} \frac{(i - \mu_i)(j - \mu_j)P_{(i,j)}}{\sigma_i\sigma_j}
	\end{equation}
	
\item Energia: soma do quadrado dos elementos da imagem. Definida por:
	\begin{equation}
		E = \sum_{i,j} P_{(i,j)}^{2}
	\end{equation}
	
\item Homogeneidade: mede a proximidade da distribuição dos elementos à diagonal da matriz. Definida por:
	\begin{equation}
		H = \sum_{i,j} \frac{P_{(i,j)}}{1+|i-j|}
	\end{equation}
\end{itemize}







	\chapter{Redes Neurais Artificiais} \label{cap:redes}

Comparado com os computadores mais avançados que existem hoje, o cérebro humano ainda se mostra muito mais poderoso e eficaz do que as máquinas. O cérebro é capaz de processar informação a uma velocidade muito maior do qualquer computador convencional e realiza com facilidade tarefas como o reconhecimento de padrões e classificação de objetos, enquanto algoritmos tradicionais falham. 

Redes neurais artificiais foram criadas com inspiração no funcionamento e na estrutura do cérebro, como uma alternativa poderosa para resolver problemas que as arquiteturas tradicionais não eram capazes de resolver eficientemente. Essas redes emulam a estrutura cerebral natural, se utilizando de elementos distintos de processamento (neurônios) que se comunicam para realizar a tarefa desejada. 

Simon Haykin\cite{Haykin} define uma rede neural como "...um processador distribuído massivamente paralelo composto por unidades simples de processamento, que possui uma propensidade natural a armazenar conhecimento experimental e torná-lo disponível para uso."

As redes neurais artificiais são utilizadas para resolver problemas como ajuste de funções, classificação de objetos e reconhecimento de padrões. Nesse capítulo, serão explicados conceitos importantes para o entendimento das redes artificiais, começando pelo seu elemento mais básico, o neurônio.

\section{Neurônios}

O neurônio é a estrutura básica do sistema nervoso central. É uma célula composta principalmente de três partes distintas: o corpo celular, os dendritos e o axônio. O corpo celular é a estrutra central do célula aonde está contido o núcleo do neurônio e são executados as suas funções vitais. Os dendritos são prolongamentos do corpo celular responsáveis por receber sinais e o axônio é uma extensão maior do corpo celular, responsável por enviar sinais a outros neurônios. Dois neurônios interagem em apenas  pontos de contato chamados sinapses. Nessas sinapses o axônio de um neurônio envia sinais para os dendritos de um segundo neurônio. A figura \ref{fig:neuronio} ilustra a estrutura básica dos neurônios.

\begin{figure}
\centering
\includegraphics[width=.6\textwidth]{neuronio}
\label{fig:neuronio}
\caption{A estrutura básica de um neurônio humano}
\end{figure}

O neurônio artificial também possui três estruturas básicas semelhantes àquelas do neurônio natural. Seus três elementos básicos são: as entradas(análogas aos dendritos), a sáida(análoga ao axônio) e um núcleo de processamento. 

O neurônio pode ter uma ou mais conexões de entrada caracterizadas por um peso $wi$ e que recebe um valor $xi$. No núcleo do neurônio artificial há um somador que calcula o valor $u$ agregado das entradas, ponderadas pelo peso correspondente. O resultado dessa soma ponderada é então deslocado de um valor escalar e em seguida submetido a uma função chamada de função de ativação, que determina a saída final do neurônio. Em suma podemos descrever a saída de um neurônio $n$ que possui $i$ entradas através das seguintes equações:

\begin{equation}
 u_n = \sum_{j=1}^i x_jw_j
\end{equation}

\begin{equation}
s(n) = f(u_n + b)
\end{equation}

Aonde $u_n$ é o valor agregado das entradas, $x_j$ é a j-ésima entrada e $w_j$ o peso da conexão associada, $s(n)$ é a saída do neurônio, $f$ é a função de ativação e $b$ o deslocamento, ou $bias$ do neurônio $n$. A figura \ref{fig:neuroartificial} ilustra a estrutura básica de um neurônio artificial.

%imagem do neurônio artifical

Claramente, a função de ativação do neurônio é o grande fator que define o seu comportamento e consequentemente sua funcionalidade. Existem dois tipos básicos de funções de ativação\cite{Haykin}: a função degrau e a função sigmóide.

O neurônio mais simples possível tem sua saída regida por uma função degrau e saída binária. Isto é, se o valor agregado $u$ de suas entradas for maior que um determinado limiar, a sua saída será 1 e a saída será 0 caso contrário. Apesar de ser capaz de resolver alguns problemas, esse tipo de neurônio tem algumas desvantagens. Por vezes, alterações sutis podem representar a diferença entre os dois valores possíveis da sáida. Além disso, um alcance limitado de valores na saída prejudica o treinamento.

A escolha mais comum de de função de ativação para a construção de redes neurais é a família de funções sigmóides. Essas funções, que tem um gráfico com formato de \textit{S}, assumem valores contínuos entre 0 e 1, o que faz com que alterações sutis na entrada representem alterações mais sutis na saída, aumentando a qualidade da informação gerada pelo neurônio. A mais comum das funções sigmóides utilizadas é a função logística, definida pela equação \ref{eq:logistica}, onde $a$ é uma constante que mede a declividade da curva.

\begin{equation}
	f(u) = \frac{1}{1 + \exp{-au}}
\label{eq:logistica}
\end{equation}

Um dos motivos que tornou a função logística uma escolha popular para os neurônios artificiais foi a sua derivação simples, uma vez que algoritmos de treinamento muitas vezes usam a derivada da função de ativação\cite{Kosabov}.

É interessante reparar que quando $a$ se aproxima do infinito, a função logística se comporta da mesma forma que a função degrau.

\begin{figure}
 \centering
\begin{subfigure}{.5\textwidth}
  \centering
  \includegraphics[width=.8\linewidth]{degrau}
	\caption{}
	\label{fig:ativacao:sub:degrau}
\end{subfigure}\
\begin{subfigure}{.5\textwidth}
  \centering
  \includegraphics[width=.8\linewidth]{logistica}
	\caption{}
	\label{fig:ativacao:sub:logistica}
\end{subfigure}
\caption{(a) Gráfico da função degrau com limiar 0. (b) Gráfico da função logística com a = 1}
\label{fig:ativacao}
\end{figure}




\section{Redes Neurais Feed-Forward}

Apesar de neurônios serem capazes de resolver alguns problemas sozinhos, o verdadeiro poder das redes neurais vem da interconexão entre os neurônios de forma a criar ligações semelhantes às sinapses dos neurônios naturais.

Existem várias arquiteturas para a formação destas redes de neurônios, porém neste trabalho a discussão será limitada àquela utilizada na implementação do programa: as chamadas redes neurais \textit{feed-forward}.

São necessárias ao menos três camadas para a construção da rede neste modelo: a primeira camada, chamada de camada de entrada, uma ou mais camadas intermediárias(ou ocultas) e a camada de saída. O papel das camadas ocultas é intermediar entre as entradas externas e a saída da rede, de forma a possibilitar que a rede extraia dados estatíticos mais significativos da sua entrada.  Uma camada é composta de um número de neurônios que agem de forma paralela. Cada neurônio de uma camada que não é a de saída está conectado apenas ao neurônios da camada seguinte, de forma que não há comunicação entre uma camada e camadas anteriores a ela. Isto é, a informação flui na rede apenas no sentido entrada-saída, como ilustrado na figura \ref{fig:feedfoward}.

\begin{figure}
\centering
\includegraphics[width=.6\textwidth]{feedforward}
\label{fig:feedfoward}
\caption{A arquitetura feed-forward. Cada neurônio se comunica com os neurônios da camada seguinte até que a saída final seja produzida. Adaptada de \cite{Haykin}}
\end{figure}

O vetor de entrada da rede é alimentado aos neurônios da primeira camada, cuja saída serve de entrada para a segunda camada e assim por diante, até que a última camada seja alimentada pela saída da camada anterior e produza a saída final da rede. 

\section{Treinamento}

Para que uma rede neural artificical possa ser utilizada, ela precisa antes aprender a realizar a tarefa para a qual foi criada. Para isso a rede precisa ter duas abilidades importantes: de aprendizado e de generalização. Aprendizado é a capacidade da rede de aproximar o comportamento das entradas fornecidas durante o treinamento, enquanto generalização é a sua capacidade de prever e operar sobre dados além do conjunto com a qual foi treinada\cite{ZhangNNSurvey}.

É necessário então que a rede passe por um processo de treinamento, onde os valores dos pesos $wi$ e os deslocamentos $b$ de cada neurônio são definidos de forma que o funcionamento da rede seja ótimo para a tarefa que se deseja realizar. Para realizar o treinamento de uma rede, o primeiro passo é definir 3 conjuntos distintos de entradas:

\begin{itemize}
	\item \textbf{Conjunto de testes:} A rede é submetida às entradas deste conjunto para que seja feita a calibração dos valores da matriz de pesos $W$ e do vetor de deslocamentos $B$.
	
	\item \textbf{Conjunto de validação:} Após uma etapa de treinamento, a rede computa os dados deste conjunto de entradas para validar os valores da matriz $W$ e do vetor $B$ escolhidos até então.
	
	\item \textbf{Conjunto de testes:} Por fim, após o treinamento, a rede é submetida aos dados deste conjunto, a fim de verificar o seu funcionamento correto.
\end{itemize}

Isto é, a cada ciclo de treinamento, a rede ajusta os seus pesos e deslocamentos de acordo com o conjunto de treinamento, e em seguida processa os dados do conjunto de validação. Se for determinado que a rede teve sucesso no processo de validação, ela fica pronta pra passar pelo conjunto de testes, aonde as sua capacidade de generalização é testada. A matriz $W$ de pesos gerada durante este processo é uma forma de representação do "conhecimento" da rede. Assim, o aprendizado não é uma característica de cada neurônio, mas sim um processo que ocorre na rede inteira como resultado do treinamento\cite{Kozabov}.

Existem duas principais maneiras de se executar o treinamento de uma rede.

\begin{itemize}
	\item \textbf{Treinamento supervisionado:} Neste paradigma os conjunto de treinamento e de validação consistem em um grupo de vetores de entrada $x$ e um gabarito de vetores de saída desejado $y$. O treinamento acontece até que o vetor de saída da rede para uma determinada entrada $x_i$ seja suficientemente próximo do vetor gabarito $y_i$ correspondete.
	
	
	\item \textbf{Treinamento não-supervisionado:} Neste paradigma a rede apenas recebe um conjunto $x$ de entradas e aprende características intrínsicas do dados apresentados a ela.
	
\end{itemize}

Uma vez definido o paradigma de treinamento, é necessário definir uma técnica para a análise do erro e alteração dos valores da matriz de pesos $W$ e do vetor $B$ de deslocamentos. Uma das técnicas mais utilizadas é a chamada de \textit{backpropagation}\cite{DeepLearning, ZhangNNSurvey}. A técnica consiste de uma análise do erro apresentado na saída da rede, que então é utilizada para se percorrer a rede no sentido contrário dos dados modificando os valores de $W$ e $B$ de forma a diminuir o erro na próxima iteração. Essa técnica é muito utilizada para sistemas de treinamento supervisionado, onde o erro é avaliado através de uma comparação da saída da rede com o gabarito fornecido. Uma vez que a os valores de $W$ e $B$ param de mudar, diz-se que a rede atingiu um estado de convergência\cite{Kozabov} e o treinamento finaliza.


\section{Classificação de padrões}

	\chapter{Trabalhos Correlatos} \label{cap:trabalhos}

Nesse capítulo serão expostas soluções para o problema apresentado no capítulo \ref{cap:intro}, acompanhadas de uma discussão sobre vantagens e desvantagens, e as razões pela escolha da solução proposta no capítulo \ref{cap:solucao}.

Podemos separar os métodos de detecção de vagas em duas grandes categorias: métodos intrusivos e não-intrusivos. Métodos intrusivos são caracterizados por exigirem instalações mais complexas, envolvendo a inserção de equipamento no asfalto do estacionamento ou em um estrutura de concreto. Os métodos não intrusivos normalmente se utilizam de equipamentos externos, que não exigem obras para a sua instalação.

Um exemplo comum de método itrusivo é o uso de sensores individuais em cada vaga do estacionamento. Diversos tipos de sensores podem ser utilizados, cada um com vantagens e desvantagens próprias \cite{idris09}. Sensores infravermelhos versáteis e de simples instalação, mas podem ser afetados por condições ambientais. Tubos pneumáticos e sensores de peso sob o asfalto têm alta cofiabilidade, mas exigem grandes obras para a instalação. 

Alguns tipos de sensores podem ser instalados de forma não intrusiva. Esses sensores podem ser afixados no teto ou em uma estrutura próxima a vaga. Sensores ultrassônicos por exemplo, pertencem a essa categoria. Eles transmitem ondas sonoras de baixa frequência e usam a energia refletida para determinar a ocupação das vagas \cite{kianpisheh2012smart}.

Intrusivos ou não, os métodos de detecção que utilizam sensores compartilham duas desvantagens: é necessário a instalação do sensor em cada uma das vagas, aumentando o custo do sistema e diminuindo sua escalabilidade, e eles não são adequados para estacionamentos descobertos, onde é impossível instalar sensores no teto e obras no asfalto rodoviário podem danificá-lo permanentemente \cite{idris09}.

Uma solução que se mostrou adequada para os estacionamentos descobertos foi o uso de \textit{softwares} de visão computacional, que analisam imagens de câmeras de vídeo para determinar a ocupação das vagas do estacionamento. Essa opção tem um custo baixo e exige apenas que sejam instaladas câmeras em pontos estratégicos do estacionamento. Uma vez que as câmeras foram instaladas e as imagens estão sendo capturadas, resta que seja executado um programa que as analise.

Para a escolha da abordagem para este trabalhos foram escolhidos os seguintes critérios principais:

\begin{itemize}
	
	\item O sistema deve poder ser instalado e começar sua execução em um estacionamento em qualquer estado, sem a necessidade de iniciar com todas as vagas descocupadas, possibilitando uma instalação a qualquer momento, sem interrupção das operações do estacionamento.
	\item O sistema deve necessitar de interação humana mínima, se limitando apenas uma rápida calibração no início de sua execução a fim de evitar erro humano.
	\item O sistema deve ser capaz de identificar a posição das vagas do estacionamento após um certo tempo de execução, e não através de entrada manual, para compensar por erros na calibração.
	\item O sistema deve, além de determinar o número de vagas livres no estacionamento, ser capaz de indicar a posição aproximada de tais vagas de forma a facilitar ainda mais a busca por uma vaga desocupada.
	
\end{itemize}

Em \cite{delibaltov2013parking}, Delibaltov \textit{et al} descrevem um método de detecção que estima um volume tridimensional para cada vaga marcada na imagem e depois se utiliza de redes neurais para determinar \textit{pixels} da imagem pertencentes a veículos. Em seguida a sistema computa a probabilidade de que uma certa vaga esteja ocupada baseado em uma função de probabilidade e os \textit{pixels} de veículo na região de interesse de cada vaga. Esse método se mostra preciso e eficaz, mas exige que o estacionamento esteja vazio e que cada vaga seja marcada individualmente no momento de execução, de forma que ele não é capaz de mapear as vagas do estacionamento ou detectar veículos estacionados irregularmente.

Bong \cite{bong2008integrated} propõe um sistema que se utiliza de subtração de imagens e o uso de detecção automática de coordenadas aproximadas das vagas para resolver este problema. Porém esse sistema exige que seja armazenada uma imagem do estacionamento completamente vazio, além de que seja criada uma imagem que marca cada vaga para o processo de inicialização do sistema.

Um sistema que se utiliza de uma técnica de classificação de histogramas de crominância em áreas de interesse e um algoritmo de que encontra pontos com \textit{features} relevantes foi proposto em \cite{true2007vacant}. Esse sistema tem a vantagem de ser relativamente invariante quanto a iluminação da imagem, por utilizar os canais de crominância para a análise. Também é bastante capaz de resolver o problema de oclusão de veículos atrás de outros na imagem de câmeras com uma angulação maior.

A solução proposta neste trabalho procura compensar alguns dos problemas presentes nos trabalhos mencionados, enquanto apresenta uma nova abordagem a ser considerada e aprimorada por trabalhos futuros.


	\chapter{Solução Proposta}\label{cap:solucao}

Uma vez que o conhecimento teórico necessário para o entendimento completo do trabalho já foi apresentado, neste capítulo será definida a solução proposta.

A solução desenvolvida consiste em um algoritmo analisa imagens de vídeo de um estacionamento descoberto e determina o número de vagas livres na imagem, além da sua localização aproximada. O sistema funciona bem em imagens de menor qualidade, mas é necessário que o vídeo adquirido seja em cores.

O trabalho se preocupa em cumprir os critérios definidos no capítulo \ref{cap:trabalhos}. Além disso, o sistema foi desenvolvido de forma que pudesse processar as imagens adquiridas de forma mais próxima possível do tempo real, causando o mínimo de atrasos para o processamento de quadros subsequentes do vídeo.

O sistema recebe como entrada um vídeo em cores capturado em um certo ângulo e um vetor de regiões de interesse no vídeo. A saída do programa a cada quadro é o número de vagas livres em cada região de interesse no vídeo.

A imagem \ref{fig:fluxograma} contém um fluxograma que mostra as etapas do processamento de cada quadro do vídeo adquirido. No decorrer deste capítulo cada um desses passo será discutido com mais detalhes.

\begin{figure}
	\centering
	\includegraphics[width=1\textwidth]{fluxograma}
	\label{fig:fluxograma}
	\caption{Um fluxograma com o funcionamento geral do programa}
	\centering
\end{figure}



\section{Aquisição}\label{sec:aquisicao}

Uma câmera montada em um poste de luz ou outro ponto similar captura as imagens utilizadas pelo programa. A câmera é montada de forma que o seu campo de visão contenha o máximo de vagas possível, porém que ainda seja possível visualizar o asfalto das vagas desocupadas e ocorra o mínimo de oclusão de veículos. A figura \ref{fig:aquisicao} mostra um quadro de uma aquisição em ângulo ideal.

\begin{figure}[!ht]
	\centering
	\includegraphics[width=8cm]{Vazio3}
	\label{fig:aquisicao}
	\caption{Um quadro de um vídeo adquirido por uma câmera do sistema}
	\centering
\end{figure}

Neste trabalho, a preocupação foi implementar um programa que analisa as imagens de uma única câmera de vídeo. Em uma aplicação no mundo real, diversas câmeras seriam instaladas para aumentar a cobertura do estacionamento. Neste caso, cada vídeo seria processado por uma cópia diferente do sistema, responsável pela área coberta pela câmera correspondente. As saídas de cada cópia do programa correspondem a ocupação das vagas em um setor diferente do estacionamento.

\section{Regiões de Interesse}\label{sec:ROIs}

No momento da instalação do programa, é necessário definir um número qualquer de regiões de interesse(\textit{Regions of interest} ou ROIs). Essas regiões determinam a área da imagem onde existem vagas. Além de determinar as regiões, deve-se informar ao programa o número de vagas existente em cada região de interesse.

As ROIs devem ser retangulares e determinadas de forma a não haver interseção entre elas, como exemplificado na figura \ref{fig:ROIs} que mostra as regiões determinadas para o quadro da figura \ref{fig:aquisicao}.

\begin{figure}
	\centering
	\includegraphics[width=8cm]{ROIs}
	\label{fig:ROIs}
	\caption{O mesmo quadro da figura \ref{fig:aquisicao}, depois de definidas as regiões de interesse, marcadas de verde.}
	\centering
\end{figure}

Depois de determinadas as regiões de interesse, cada uma delas é dividida em um número igual de seções verticais como ilustrado na figura \ref{fig:secoesVerticais}. Como as vagas não são determinadas individualmente no momento da instalação do programa, é necessário que seja feita alguma divisão das ROIs. Cada uma destas seções verticais é classificada separadamente em uma etapa futura do processamento. O resultado da classificação de cada seção de uma ROI determina um vetor $v$ de $n$ elementos, onde $n$ é o número de seções em que a região foi dividida e cada elemento indica a ocupação da vaga, sendo o valor $1$ correspondente a uma vaga ocupada e o valor $2$ correspondente a uma vaga livre. Através da análise deste vetor é que o programa determina o número de vagas livres em cada região. Munido do número de vagas que cada região contém e um valor aproximado do número de seções que um carro ocupa, o programa estima quantas vagas estão ocupadas, e encontra o número de vagas livres através de simples subtração e a posição aproximada destas vagas através da posição das seções livres no vetor.

\begin{figure}
	\centering
	\includegraphics[width=8cm]{Secoes}
	\label{fig:secoesVerticais}
	\caption{As ROIs determinadas na figura \ref{fig:ROIs} separadas em trinta seções verticais que ainda não foram classificadas. Nesta etapa os vetores $v$ de cada região é composto de trinta valores $2$.}
	\centering
\end{figure}


Uma vez que as regiões de interesse foram definidas e divididas em seções, a etapa de calibração é finalizada e o programa pode iniciar o processo de determinação da ocupação das vagas. 

\section{Classificação das seções} \label{sec:classificacao}

O primeiro passo na execução do programa depois da calibração é a classificação de cada uma das sessões verticais criadas na imagem. Essa classificação é feita assim que a execução começa, no primeiro quadro do vídeo e é repetida a cada segundo de execução do programa. Além disso, quando se detecta movimento em uma das ROIs as sessões interceptadas pelo movimento detectada são classificadas. O processo de detecção de movimento e determinação das sessões a serem classificadas excepcionalmente é detelhado na seção \ref{sec:sub:regioesmovimento}.

Para que seja feita a extração das caracterísiticas e a classificação de uma seção, uma imagem $I_s$ é extraída do quadro de vídeo original capturado.Essa imagem corresponde à fração do quadro contida dentro dos limites da seção a ser analisada. A extração de $I_s$ consiste simplesmente em copiar os valores da matriz que representa o quadro que estavam dentro das bordas da seção em uma nova matriz de dimensões iguais às da seção. A figura \ref{fig:imgSecao} mostra um exemplo de uma $I_s$. Cada uma dessas imagens é submetida separadamente ao processo de classificação.

\begin{figure}
	\centering
	\includegraphics[height=5cm]{imgSecao}
	\label{fig:imgSecao}
	\caption{Uma imagem correspondente a uma das seções verticais definidas.}
	\centering
\end{figure}


\subsection{Extração de características}

O processo de classificação de uma seção começa com a extração das características que definem o padrão de cada classe e a criação do vetor de entrada da rede neural artifical. Dois conjuntos de características principais são extraídos: as quatro medidas extraídas da GLCM da imagem apresentadas na seção \ref{sec:GLCM} e dois valores que descrevem a crominância azul e vermelha da seção.

Para a extração da GLCM é preciso primeiro obter uma imagem de níveis de cinza que represente $I_s$. Para isso, a imagem é convertida para o espaço $YCbCr$ como descrito na seção \ref{sec:sub:ycbcr}. A imagem referente ao canal $Y$ resultante é uma imagem de intensidade e por isso é utilizada para o cálculo da GLCM. Como mencionado na seção \ref{sec:GLCM}, a construção é feita com base na vizinhança a direita de cada \textit{pixel} da imagem. Cada valor dentre os 255 possíveis é dividido dentro de 8 níveis distintos, e sempre que um valor presente em nível $N$ aparece a direita de um valor de um nível $M$, o elemento $(N,M)$ da GLCM é incrementado. A figura \ref{fig:GLCM} mostra esse processo.

Uma vez construída a GLCM referente a sessão, quatro medidas estatísticas são extraídas: contraste, correlação, energia e homogeneidade. Essas medidas são calculadas pelas equações \ref{eq:Contraste}, \ref{eq:Correlacao},\ref{eq:Energia} e \ref{eq:Homo} respectivamente. Esses valores formam um vetor $g = (contraste, correlação, energia, homogeneidade)^T$ que é parte da entrada final da rede neural artificial.

Os outros dois \textit{features} extraídos de cada sessão vêm dos canais $Cb$ e $Cr$ de $I_s$. A média de valores da matriz de cada canal é calculada através da equação \ref{eq:media}. Onde $N$ é o número total de \textit{pixels} de $I_s$ e $v_i$ é o valor do \textit{i-ésimo pixel}. Esses valores então formam o vetor $c = (M_{Cb}, M_{Cr})^T$.

\begin{equation}
	M = \frac{\sum_i=1^N v_i}{N}
\label{eq:media}
\end{equation}


Esses \textit{features} foram escolhidos por terem se mostrado suficientemente descritivos e distintivos após uma análise de um conjunto de 90 imagens. As imagens \ref{fig:histContraste},\ref{fig:histCorrelacao},\ref{fig:histEnergia},\ref{fig:histHomo},\ref{fig:histCb} e \ref{fig:histCr} mostram histogramas que descrevem a distribuição dos \textit{features} no conjunto de imagens. Em cada um dos gráficos, o eixo $x$ correspondente a valores obtidos para caracterísitica em questão e o eixo $y$ indica o número de imagens que exibiram aquele valor. As barras azuis são referentes as imagens de carros ou vagas ocupadas e as barras vermelhas referentes a vagas vazias.

\begin{figure}
	\centering
	\includegraphics[width=11cm]{Contraste}
	\label{fig:histContraste}
	\caption{O histograma referente aos valores de contraste no conjunto analisado. Apesar de haver bastante sobreposição dos valores ainda é possível ver que há pouco ou nenhuma ocorrência de vagas após um certo valor.}
	\centering
\end{figure}

\begin{figure}
	\centering
	\includegraphics[width=11cm]{Correlacoes}
	\label{fig:histCorrelacao}
	\caption{O histograma referente aos valores de correlacao no conjunto analisado. Aqui é possível ver um limiar inferior para a classe dos carros. Valores abaixo deste limiar provavelmente são vagas livres.}
	\centering
\end{figure}

\begin{figure}
	\centering
	\includegraphics[width=11cm]{Energia}
	\label{fig:histEnergia}
	\caption{O histograma referente aos valores de energia no conjunto analisado. Há um ponto de divisão entre as duas classes, com pouca sobreposição de valores entre as classes.}
	\centering
\end{figure}

\begin{figure}
	\centering
	\includegraphics[width=11cm]{Homogeneidade}
	\label{fig:histHomo}
	\caption{O histograma referente aos valores de homogeneidade no conjunto analisado. Esse gráfico mostra mais sobreposição do que os anteriores, mas ainda contém bastante informação sobre a classe das vagas. }
	\centering
\end{figure}

\begin{figure}
	\centering
	\includegraphics[width=11cm]{CrominanciaAzul}
	\label{fig:histCb}
	\caption{O histograma referente aos valores médios de crominância azul no conjunto analisado. Apesar de haver alta variedade nos valores encontrados nas imagens de carros, as imagens de faixa possuem valores médios em uma faixa estreita. }
	\centering
\end{figure}

\begin{figure}
	\centering
	\includegraphics[width=11cm]{CrominanciaVermelha}
	\label{fig:histCr}
	\caption{O histograma referente aos valores médios de crominância vermelha no conjunto analisado, mostrando comportamento semelhante ao histograma de de valores médios de crominância azul. }
	\centering
\end{figure}


\subsection{Classificação da rede neural artificial}

\subsection{Ajuste gaussiano dos vizinhos}

\subsection{Sistema de votação}


\section{Extração do movimento}

\subsection{Fluxo Óptico}

\subsection{Regiões de movimento}\label{sec:sub:regioesmovimento}






	\chapter{Resultados Experimentais}\label{cap:results}

\section{Aquisição de dados}

Para os experimentos realizados neste trabalho foi capturado um conjunto de vídeos no estacionamento do pavilhão João Calmon na Universidade de Brasília. Foram feitas duas seções de filmagens contínua, aonde três veículos se deslocavam pelo estacionamento e ocupavam vagas escolhidas arbitrariamente. O primeiro vídeo capturado foi utilizado para o treinamento da rede. O segundo vídeo foi dividido em oito vídeos de menor duração sobre os quais foram realizados os testes.

As filmagens foram realizadas por um drone modelo \textit{Yuneec Typhoon Q500+} (Figura \ref{fig:drone}). O drone foi controlado para que pairasse no ar em uma altura semelhante a de um poste de luz, a fim de capturar imagens de forma mais próxima possível de uma câmera de vídeo instalada em um poste de luz.

\begin{figure}
\centering
\includegraphics[width=8cm]{drone}
\caption{O drone utilizado para a gravação dos vídeos de teste.}
\label{fig:drone}
\centering
\end{figure}

Para a aquisição das imagens utilizadas para o treinamento da rede, o primeiro vídeo passou por um processo semelhante ao processo de configuração inicial do programa final descrito no Capítulo \ref{cap:solucao}. Duas áreas de interesse foram determinadas de forma idêntica ao processo de escolha de ROIs do \textit{DVE}. Em seguida cada ROI foi dividida em trinta seções verticais. Três quadros do vídeo foram escolhidos de forma aleatória. As imagens correspondentes a cada seção de cada quadro escolhido foram extraídas em arquivos separados. Dessa maneira, as imagens utilizadas para o treinamento da rede neural foram construídas da mesma maneira que as imagens que seriam alimentadas a rede para classificação no futuro.

\section{Treinamento da rede neural artificial}
Uma vez adquiridas as imagens a serem utilizadas para o treinamento, cada uma das imagens foi separada manualmente em uma das duas categorias:veículo ou vaga vazia. Ao todo foram utilizadas $57$ imagens classificadas como veículo (categoria 1) e $59$ imagens classificadas como vaga vazia (categoria 2) totalizando $116$ imagens. Foram extraídas as características de cada imagem da forma descrita na Seção \ref{sec:extracao}. Os vetores $x_s$ obtidos foram então concatenados de forma a criar duas matrizes. A primeira de dimensões $6\times 57$ representava as características das imagens de carros e a segunda de dimensões $6\times 59$ as características das imagens de vagas desocupadas.

\begin{figure}
\centering
\begin{subfigure}{.1\textwidth}
  \centering
  \includegraphics[width=.8\linewidth, height=5cm]{ocupada}
  \caption{}
  \label{fig:exemploRede:sub:ocupada}
\end{subfigure}%
\begin{subfigure}{.1\textwidth}
  \centering
  \includegraphics[width=.8\linewidth, height=5cm]{desocupada}
  \caption{}
  \label{fig:exemploRede:sub:desocupada}
\end{subfigure}
\centering
\caption{(a)Um exemplo de imagem classificada manualmente na classe 1;(b) Um exemplo de imagem classificada manualmente na classe 2.}
\label{fig:exemploRede}
\end{figure}

Para que pudesse ser feito o treinamento supervisionado da rede neural, duas matrizes de alvos foram construídas. Uma com dimensões $2\times 57$ e a outra com dimensões $2\times 59$. A primeira matriz com todas as colunas iguais a $\begin{psmallmatrix}1\\0\end{psmallmatrix}$ e a segunda com as colunas da forma $\begin{psmallmatrix}0\\1\end{psmallmatrix}$. Essas matrizes vão compor o gabarito utilizado para o treinamento supervisionado da rede.

As matrizes  de características de cada classe foram então concatenadas formando uma matriz de entrada $In_{6\times 116}$. O mesmo foi feito com as matrizes do gabarito, resultando na matriz $Target_{2\times 116}$. Em seguida as colunas dessas duas matrizes foram reorganizadas aleatoriamente, de forma que as mudanças feitas em $In$ fossem refletidas em $Out$, garantindo que não fosse perdida a relação entre as colunas de mesmo índice das matrizes. O resultado final deste processo é que cada coluna de $Target_{2\times 116}$ indica a classe da coluna correspondente de $In_{6\times 116}$.

Uma vez criadas essas duas matrizes, as entradas e seus respectivos alvos são separados em três conjuntos: o conjunto de treinamento, de validação e de testes. A divisão é feita de forma que $70\%$ das entradas são designadas ao conjunto de treinamento, $15\%$ designadas ao conjunto de validação e os $15\%$ restantes ao conjunto de testes. O conjunto de treinamento então é composto por duas matriz $Ti_{6\times 81}$ e $To_{2\times 81}$ que são iguais as primeira $81$ colunas de $In_{6\times 116}$ e $Target_{2\times 116}$ e representam os vetores descritores das entradas e seus gabaritos respectivamente. Os outros dois conjuntos são construídos de forma similar utilizando. O conjunto de validação é composto por matrizes de $18$ colunas e o de teste por matrizes de $17$ colunas.


A rede neural utilizada foi uma rede \textit{feed-forward} com três camadas. A camada oculta possui $15$ neurônios com função de ativação logística (Equação\ref{eq:logistica}) e a camada de saída $2$ neurônios com função de ativação \textit{softmax} (Equação \ref{eq:softmax}). A rede é submetida a um treinamento supervisionado com um limite de $1000$ iterações. Quando a situação de convergência é atingida, o treinamento para e o resultado é a rede final utilizada.


\section{Resultados Obtidos}

Para determinar a capacidade do \textit{DVE} de determinar se as seções verticais estão ocupadas ou livres, os resultados obtidos pelo programa foram comparados com os resultados que eram esperados de observadores humanos. Foram apresentados a três observadores, que chamaremos pelas iniciais F, M e P, um conjunto de oito vídeos que mostravam veículos estacionando ou saindo de vagas em um estacionamento descoberto. As áreas de interesse do vídeo foram definidas previamente e divididas em $30$ seções verticais. Dessa forma, todos os observadores e o \textit{DVE} analisaram seções verticais idênticas. 

Cada um dos observadores receberam as seguintes instruções:

\begin{itemize}
  \item As regiões de interesse e as seções são numeradas como indicado na Figura \ref{fig:instrucao}.
	\item Uma seção ocupada é aquela cuja maior parte de sua área está ocupada por um veículo.
	\item No momento inicial do vídeo ($0$ segundos), indique quais seções verticais estão ocupadas através do número das ROI e das seções, as outras serão assumidas como livres. Indique também o número de vagas ocupadas.
	\item Se a qualquer momento o estado de ocupação de uma seção mudar, indique o tempo da mudança, a seção onde ocorreu a mudança e a natureza da mudança (ocupada ou liberada).
\end{itemize}

\begin{figure}
\centering
\includegraphics[width=8cm]{instrucoes}
\centering
\caption{Figura mostrando a divisão das ROIs e numeração das seções nos vídeos mostrados para os observadores humanos.}
\label{fig:instrucao}
\end{figure}

Cada observador assistiu aos vídeos sozinho, sem interferência externa e sem conhecimento dos resultados do programa. De fato, a fim de evitar interferência, o \textit{DVE} só foi executado sobre os casos de teste após todos os observadores escolhidos terem entregado os resultados que obteram.

Para os testes apresentados a seguir, serão observadas somente aquelas seções que o \textit{DVE} ou os observadores determinaram como ocupadas. As demais seções serão consideradas livres. Chamaremos de um \textit{acerto} sempre que em um determinado segundo $t$, o programa e um observador concordam quanto a ocupação de uma seção vertical. Sendo assim, o programa pode atingir um máximo de $60$ acertos por segundo. A indicação da ocupação das seções pelos observadores não foi definida a cada segundo, mas considera-se que entre duas acusações de mudança de estado a ocupação das seções permanece a mesma. Sendo assim, podemos utilizar esses momentos para definir a ocupação em cada segundo do vídeo. Por exemplo, se um observador disser que a seção $10$ estava ocupada no momento inicial do vídeo e depois indicar que a mesma sessão foi liberada aos $8s$, a seção será considerada ocupada durante todos os momentos deste intervalo. 

Além da capacidade de classificar cada seção, a precisão do \textit{DVE} em determinar o número de vagas ocupadas a cada momento do vídeo também foi analisada. Para este teste, os observadores foram instruídos a informar o número de vagas ocupadas no início do vídeo. Sempre que considerassem que houve uma mudança neste número, deveriam informar o momento da mudança e o novo número de vagas ocupadas. Para esta análise, um \textit{acerto} é considerado a cada segundo que o programa indica o mesmo número de vagas ocupadas que os observadores. Para a determinação das vagas o programa foi executado no modo complexo de mapeamento.

Nas seções seguintes serão apresentados cada um dos vídeos utilizados para os testes. Cada caso iniciará com uma breve descrição do movimento dos veículos no vídeo, a duração do vídeo e o número de acertos possíveis para as seções verticais, seguidos de cinco tabelas, uma para cada observador e duas para o \textit{DVE}. As tabelas indicam o tempo onde foram acusadas mudanças de estado nas seções e no número de vagas ocupadas. A primeira linha da tabela indica as seções ocupadas no momento inicial do vídeo e cada linha subsequente indica um momento em segundos quando houve mudança de estado de pelo menos uma seção ou vaga, a ROI e o número das seções modificadas e o número de vagas ocupadas naquele momento no tempo.  Finalmente, será calculada uma taxa de acerto que compara o desempenho do programa na classificação a cada observador e uma taxa final de acerto média para o caso de teste. A taxa de acerto para cada observador é calculada pela razão entre o número de acertos do programa e o número de acertos possíveis para o caso de teste e a taxa de acerto média é a média aritmética entre estes valores. Para que o leitor possa comparar o desempenho do \textit{DVE} com suas próprias impressões, uma imagem dos momentos iniciais e finais de cada vídeo será mostrada junto de cada análise.

A leitura das tabelas que representam as taxas de acerto devem ser feitas com um certo cuidado. Por causa da grande quantidade de seções e de possíveis acertos, um erro na classificação de uma seção representa uma mudança pequena na taxa de acerto. Em cada caso de testes há um grande número de seções aonde não há movimento algum, que são classificadas corretamente como vazias, elevando a taxa de acertos do teste. Enquanto esses acertos são relevantes, a classificação correta das seções onde há movimento de veículos representa uma diferença muito maior no funcionamento do \textit{DVE}.

Por outro lado, a taxas de acerto das ocupações das vagas são muito mais afetadas por erros. Se o \textit{DVE} indicar uma vaga ocupada a mais que um observador por apenas $1$ segundo, um erro quase insignificante em termos humanos, isso pode representar uma diminuição de até $10\%$ na taxa de acerto. 

Por isso, as taxas de acerto não são fatores exclusivos na determinação da eficácia do programa. Assim, uma análise rápida do comportamento do \textit{DVE} em cada caso de testes será apresentada após as tabelas.

\subsection{Vídeo 1}

No primeiro caso de testes, o vídeo começa com um estacionamento vazio. Depois de alguns segundos um único veículo de cor branca entra na cena pela direita e estaciona em uma vaga na parte inferior da imagem. O vídeo tem $15$ segundos de duração, totalizando $900$ acertos possíveis.

\begin{figure}[H]
\centering
\begin{subfigure}{.5\textwidth}
\centering
\includegraphics[width=.8\linewidth]{Video1Inicio}
\caption{}
\end{subfigure}\
\begin{subfigure}{.5\textwidth}
\centering
\includegraphics[width=.8\linewidth]{Video1Fim}
\caption{}
\end{subfigure}
\centering
\caption{(a) O momento inicial do vídeo 1; (b) O momento final do vídeo 1.}%
\label{}%
\end{figure}

\begin{table}[H]
\begin{center}
\begin{tabular}{|c||c||c|}
\hline
\multicolumn{3}{|c|}{Observador F}  \\ \hline \hline
Tempo(s) & Acontecimento & Vagas ocupadas\\ \hline
0 & Nenhuma seção ocupada & 0 \\ \hline
10 & ROI 2: Seções 18 e 19 ocupadas. & 1 \\
\hline
\end{tabular}
\caption{Tabela com os resultados do observador F referentes ao Vídeo 1.}
\label{tab:video1F}
\end{center}
\end{table}

\begin{table}[H]
\begin{center}
\begin{tabular}{|c||c||c|}
\hline
\multicolumn{3}{|c|}{Observador P}  \\ \hline \hline
Tempo(s) & Acontecimento & Vagas ocupadas\\ \hline
0 & Nenhuma seção ocupada & 0 \\ \hline
10 & ROI 2: Seções 18 e 19 ocupadas. & 1 \\
\hline
\end{tabular}
\caption{Tabela com os resultados do observador P referentes ao Vídeo 1.}
\label{tab:video1F}
\end{center}
\end{table}

\begin{table}[H]
\begin{center}
\begin{tabular}{|c||c||c|}
\hline
\multicolumn{3}{|c|}{Observador M}  \\ \hline \hline
Tempo(s) & Acontecimento & Vagas ocupadas\\ \hline
0 & Nenhuma seção ocupada & 0 \\ \hline
9 & ROI 2: Seções 18 e 19 ocupadas. & 1 \\
\hline
\end{tabular}
\caption{Tabela com os resultados do observador M referentes ao Vídeo 1.}
\label{tab:video1M}
\end{center}
\end{table}

\begin{table}[H]
\begin{center}
\begin{tabular}{|c||c||c|}
\hline
\multicolumn{3}{|c|}{DVE}  \\ \hline \hline
Tempo(s) & Acontecimento & Vagas Ocupadas\\ \hline
0 & ROI 2:Seção 3 ocupada. & 1 \\ \hline
10 & ROI 2: Seções 18 e 19 ocupadas. & 2\\ \hline
14 & ROI 2: Seção 3 liberada. & 1\\
\hline
\end{tabular}
\caption{Tabela com os resultados do \textit{DVE} referentes ao Vídeo 1.}
\label{tab:video1P}
\end{center}
\end{table}

\begin{table}[H]
\begin{center}
\begin{tabular}{|c||c||c|}
\hline
\multicolumn{3}{|c|}{Acertos - seções}  \\ \hline \hline
Observador & Acertos& Taxa de acertos \\ \hline
F & 886 & 98,44\% \\  \hline
P & 886 & 98,44\% \\ \hline
M & 884 & 98,22\% \\ \hline
Média & 885,3 & 98,37\% \\
\hline
\hline
\multicolumn{3}{|c|}{Acertos - vagas}  \\ \hline \hline
Observador & Acertos & Taxa de acertos \\ \hline
F & 1 & 7\% \\  \hline
P & 1 & 7\% \\ \hline
M & 1 & 7\% \\ \hline
Média & 1 & 7\% \\
\hline
\end{tabular}
\caption{Tabela com o número de acertos e a taxa de acertos do \textit{DVE} nos dois critérios avaliados no Vídeo 1.}
\label{tab:rvideo1}
\end{center}
\end{table}


Neste caso de testes, o programa classifica errôneamente a seção $3$ da ROI $2$ durante quase toda a duração do vídeo. Como o erro se inicia logo no primeiro momento do vídeo, esta seção é definida como uma vaga que conta como ocupada enquanto o erro de classificação persiste, fazendo com que o programa indique um número de vagas ocupadas errado durante toda a sua execução. Esse caso ilustra uma possível consequência de um erro de classificação. Apesar do erro, o reconhecimento da vaga ocupada pelo veículo nas seções $18$ e $19$ ocorre sem problemas.

\subsection{Vídeo 2}

Neste vídeo, um carro branco está estacionado no conjunto inferior de vagas. Um veículo cinza entra pela direita e estaciona no conjunto superior de vagas. O vídeo tem uma duração de $10$ segundos, totalizando $600$ acertos possíveis.

\begin{figure}[H]
\centering
\begin{subfigure}{.5\textwidth}
\centering
\includegraphics[width=.8\linewidth]{Video2Inicio}
\caption{}
\end{subfigure}\
\begin{subfigure}{.5\textwidth}
\centering
\includegraphics[width=.8\linewidth]{Video2Fim}
\caption{}
\end{subfigure}
\centering
\caption{(a) O momento inicial do vídeo 2; (b) O momento final do vídeo 2.}%
\label{}%
\end{figure}

\begin{table}[H]
\begin{center}
\begin{tabular}{|c||c||c|}
\hline
\multicolumn{3}{|c|}{Observador F}  \\ \hline \hline
Tempo(s) & Acontecimento & Vagas Ocupadas\\ \hline
0 & ROI 2: Seções 19 e 20 ocupadas & 1 \\ \hline
6 & ROI 1: Seções 21,22 e 23 ocupadas. & 2 \\
\hline
\end{tabular}
\end{center}
\caption{Tabela com os resultados do observador F referentes ao Vídeo 2.}
\label{tab:video2F}
\end{table}

\begin{table}[H]
\begin{center}
\begin{tabular}{|c||c||c|}
\hline
\multicolumn{3}{|c|}{Observador P}  \\ \hline \hline
Tempo(s) & Acontecimento & Vagas Ocupadas\\ \hline
0 & ROI 2: Seções 19 e 20 ocupadas  & 1\\ \hline
6 & ROI 1: Seções 20,21,22 ocupadas. & 2 \\
\hline
\end{tabular}
\end{center}
\caption{Tabela com os resultados do observador P referentes ao Vídeo 2.}
\label{tab:video2M}
\end{table}


\begin{table}[H]
\begin{center}
\begin{tabular}{|c||c||c|}
\hline
\multicolumn{3}{|c|}{Observador M}  \\ \hline \hline
Tempo(s) & Acontecimento & Vagas Ocupadas\\ \hline
0 & ROI 2: Seções 19 e 20 ocupadas & 1\\ \hline
6 & ROI 1: Seções 21,22 e 23 ocupadas. & 2\\
\hline
\end{tabular}
\end{center}
\caption{Tabela com os resultados do observador M referentes ao Vídeo 2.}
\label{tab:video2P}
\end{table}


\begin{table}[H]
\begin{center}
\begin{tabular}{|c||c||c|}
\hline
\multicolumn{3}{|c|}{DVE}  \\ \hline \hline
Tempo(s) & Acontecimento & Vagas Ocupadas\\ \hline
0 & ROI 2: Seções 18,19 e 20 ocupadas. & 1 \\ \hline
6 & ROI 1: Seções 21 e 22 ocupadas. & 2 \\
\hline
\end{tabular}
\end{center}
\caption{Tabela com os resultados do \textit{DVE} referentes ao Vídeo 2.}
\label{tab:video2}
\end{table}


\begin{table}[H]
\begin{center}
\begin{tabular}{|c||c||c|}
\hline
\multicolumn{3}{|c|}{Acertos - seções}  \\ \hline \hline
Observador & Acertos & Taxa de acertos \\ \hline
F & 586 & 97,66\% \\  \hline
P & 586 & 97,66\% \\ \hline
M & 586 & 97,66\% \\ \hline
Média & 586 & 97,66\% \\
\hline
\hline
\multicolumn{3}{|c|}{Acertos - vagas}  \\ \hline \hline
Observador & Acertos & Taxa de acertos \\ \hline
F & 10 & 100\% \\  \hline
P & 10 & 100\% \\ \hline
M & 10 & 100\% \\ \hline
Média & 10 & 100\% \\
\hline
\end{tabular}

\end{center}
\caption{Tabela com o número de acertos e a taxa de acertos do \textit{DVE} nos dois critérios avaliados no Vídeo 2.}
\label{tab:rvideo2}
\end{table}




Neste caso de testes, houve discordância entre os observadores sobre as seções ocupadas pelo carro cinza na parte superior do vídeo. Porém o \textit{DVE} classificou como ocupadas as seções onde houve consenso entre os observadores.

O programa também classificou errôneamente a seção $18$ do ROI $2$ no momento inicial do vídeo. Novamente o \textit{DVE} acerta nas seções onde há consenso entre os observadores. O erro é mais aceitável que o erro que ocorreu no vídeo $1$, pois a determinação da vaga ainda ocorre de maneira que a sua ocupação é reconhecida.

\subsection{Vídeo 3}

Neste vídeo, dois carros se encontram no estacionamento. Um veículo amarelo entra pela direita e estaciona na seção superior das vagas. O veículo branco sai pela parte de baixo da tela. O vídeo tem duração de $20$ segundos totalizando $1200$ acertos possíveis.

\begin{figure}[H]
\centering
\begin{subfigure}{.5\textwidth}
\centering
\includegraphics[width=.8\linewidth]{Video3Inicio}
\caption{}
\end{subfigure}\
\begin{subfigure}{.5\textwidth}
\centering
\includegraphics[width=.8\linewidth]{Video3Fim}
\caption{}
\end{subfigure}
\centering
\caption{(a) O momento inicial do vídeo 3; (b) O momento final do vídeo 3.}%
\label{}%
\end{figure}


\begin{table}[H]
\begin{center}
\begin{tabular}{|c||c||c|}
\hline
\multicolumn{3}{|c|}{Observador F}  \\ \hline \hline
Tempo(s) & Acontecimento & Vagas Ocupadas\\ \hline
0 & ROI 1: Seções 22 e 23 ocupadas. & 2 \\
 & ROI 2: 19,20 e 21 ocupadas & \\ \hline
8 & ROI 1: Seções 16 e 17 ocupadas. & 3\\ \hline
13 & ROI 2: Seções 18 e 19 liberadas. & 2\\
\hline
\end{tabular}
\end{center}
\caption{Tabela com os resultados do observador F referentes ao Vídeo 3.}
\label{tab:video3F}
\end{table}

\begin{table}[H]
\begin{center}
\begin{tabular}{|c||c||c|}
\hline
\multicolumn{3}{|c|}{Observador P}  \\ \hline \hline
Tempo(s) & Acontecimento & Vagas Ocupadas\\ \hline
0 & ROI 1: Seções 22 e 23 ocupadas. & 2 \\
 & ROI 2: 20 e 21 ocupadas &  \\ \hline
9 & ROI 1: Seções 15,16 e 17 ocupadas. & 3 \\
\hline
\end{tabular}
\end{center}
\caption{Tabela com os resultados do observador P referentes ao Vídeo 3.}
\label{tab:video3P}
\end{table}

\begin{table}[H]
\begin{center}
\begin{tabular}{|c||c||c|}
\hline
\multicolumn{3}{|c|}{Observador M}  \\ \hline \hline
Tempo(s) & Acontecimento & Vagas Ocupadas\\ \hline
0 & ROI 1: Seções 22 e 23 ocupadas. & 2 \\
 & ROI 2: 20 e 21 ocupadas & \\ \hline
8 & ROI 1: Seções 15,16 e 17 ocupadas. & 3 \\ \hline
14 & ROI 2: Seções 17,18 e 19 liberadas & 2\\
\hline
\end{tabular}
\end{center}
\caption{Tabela com os resultados do observador M referentes ao Vídeo 3.}
\label{tab:video3M}
\end{table}

\begin{table}[H]
\begin{center}
\begin{tabular}{|c||c||c|}
\hline
\multicolumn{3}{|c|}{DVE}  \\ \hline \hline
Tempo(s) & Acontecimento & Vagas Ocupadas\\ \hline
0 & ROI 1: Seções 22 e 23 ocupadas. & 2 \\
 & ROI 2: 19,20 e 21 ocupadas &  \\ \hline
1 & ROI 1: Seção 23 liberada. & 2 \\ \hline
2 & ROI 1: Seção 23 ocupada. & 2 \\ \hline
4 & ROI 1: Seção 23 liberada. & 2 \\ \hline
8 & ROI 1: Seções 16 e 17 ocupadas. & 3 \\ \hline
9 & ROI 2: Seção 21 liberada. & 3 \\ 
 & ROI 2: Seções 18 e 24 ocupadas &  \\ \hline
10 & Sem acontecimentos & 2 \\ \hline
11 & ROI 1: Seção 20 ocupada. & 3 \\ 
 & ROI 2: Seção 24 liberada. \\ \hline
13 & ROI 1: Seção 20 liberada. & 2 \\
 & ROI 2: Seção 18,19 e 20 liberada. & \\ \hline
16 & ROI 1: Seção 20 ocupada. & 2 \\ \hline
19 & ROI 1: Seção 20 liberada. & 2 \\
\hline
\end{tabular}
\end{center}
\caption{Tabela com os resultados do \textit{DVE} referentes ao Vídeo 3.}
\label{tab:video3}
\end{table}

\begin{table}[H]
\begin{center}
\begin{tabular}{|c||c||c|}
\hline
Observador & Acertos & Taxa de acertos \\ \hline
F & 1165 & 97,08\% \\  \hline
P & 1113 & 92,75\% \\ \hline
M & 1136 & 94,66\% \\ \hline
Média & 1138 & 94,83\% \\
\hline
\hline
\multicolumn{3}{|c|}{Acertos - vagas}  \\ \hline \hline
Observador & Acertos & Taxa de acertos \\ \hline
F & 19 & 95\% \\  \hline
P & 11 & 55\% \\ \hline
M & 18 & 90\% \\ \hline
Média & 16 & 80\% \\
\hline
\end{tabular}
\end{center}
\caption{Tabela com o número de acertos e a taxa de acertos do \textit{DVE} nos dois critérios avaliados no Vídeo 3.}
\label{tab:rvideo3}
\end{table}



Esse caso de teste possui alguns problemas causados principalmente por causa da movimentação do \textit{drone} no momento da gravação. Essa movimentação faz com que partes da imagem passem a ocupar seções diferentes, criando algumas incosistências. Repare que por volta dos $14s$ dois dos observadores disseram que a seção $18$ da ROI $2$ foi liberada, apesar de nunca terem acusado que a seção estava ocupada anteriormente. Essa inconsistência ocorre porque entre o início do vídeo esse momento, o \textit{drone} utilizado para a gravação se movimenta e faz com que o veículo que estava ocupando as  seções $19$ e $20$ ocupe as seções $18$ e $19$. Esse problema acontece em outros casos de teste em menor proporção.

O teste sobre esse vídeo ainda possui resultados interessantes e úteis. Uma vantagem do uso de um sistema automatizado se mostra neste caso de teste. O observador \textit{P} não percebeu que um veículo saia da tela por volta dos $14$ segundos e por isso não acusou a mudança de estado de nenhuma seção nesse momento. Uma falha que não ocorre no programa.

Através deste caso também percebe-se que o \textit{DVE} tem dificuldade em classificar corretamente as seções ocupadas pelo veículo cinza na ROI $1$, evidenciada principalmente pela flutuação da classificação das seções $23$ e $20$.

Mesmo com os problemas, o mapeamento e determinação das ocupações das vagas mostra uma alta taxa de acerto. Neste caso de teste, a determinação inicial das vagas funciona a favor do programa. Apesar de ter dificuldades em classificar o carro cinza, o \textit{DVE} identifica uma das seções ocupadas pelo veículo e determina que esta seção pertence a uma vaga. Por isso, o número de vagas ocupadas se mantém correto durante a execução do teste. Uma flutuação na classificação da seção causa um erro por um segundo, mas que é rapidamente corrigido. Além disso o sistema de mapeamento se mostrou eficaz e seguro. Quando o deslocamento do drone ocorre, diversos movimentos são detectados no vídeo, causando a marcação de várias vagas. Apesar disso, nenhuma seção fica marcada em duas vagas.



\subsection{Vídeo 4}

Neste vídeo, o veículo branco entra em cena pela parte superior da tela e estaciona entre os outros dois carros. O carro cinza sai da vaga que ocupava e estaciona em uma na área inferior. O vídeo possui uma duração de $30$ segundos, portanto $1800$ possíveis acertos.

\begin{figure}[H]
\centering
\begin{subfigure}{.5\textwidth}
\centering
\includegraphics[width=.8\linewidth]{Video4Inicio}
\caption{}
\end{subfigure}\
\begin{subfigure}{.5\textwidth}
\centering
\includegraphics[width=.8\linewidth]{Video4Fim}
\caption{}
\end{subfigure}
\centering
\caption{(a) O momento inicial do vídeo 4; (b) O momento final do vídeo 4.}%
\label{}%
\end{figure}

\begin{table}[H]
\begin{center}
\begin{tabular}{|c||c||c|}
\hline
\multicolumn{3}{|c|}{Observador F}  \\ \hline \hline
Tempo(s) & Acontecimento & Vagas Ocupadas\\ \hline
0 & ROI 1: Seções 15,16,21 e 22 ocupadas. & 2 \\ \hline
7 & ROI 1: Seções 19 e 20 ocupadas. & 3 \\ \hline
16 & ROI 1: Seções 21 e 22 liberadas. & 2 \\ \hline
28 & ROI 2: Seções 12 e 13 ocupadas. & 3 \\
\hline
\end{tabular}
\end{center}
\caption{Tabela com os resultados do observador F referentes ao Vídeo 4.}
\label{tab:video4F}
\end{table}

\begin{table}[H]
\begin{center}
\begin{tabular}{|c||c||c|}
\hline
\multicolumn{3}{|c|}{Observador P}  \\ \hline \hline
Tempo(s) & Acontecimento & Vagas Ocupadas\\ \hline
0 & ROI 1: Seções 15,16,17,20 e 21 ocupadas. & 2 \\ \hline
8 & ROI 1: Seções 18 e 19 ocupadas. & 3 \\ \hline
18 & ROI 1: Seções 21 e 22 liberadas & 2 \\ \hline
29 & ROI 2: 12,13 e 14 ocupadas & 3 \\
\hline
\end{tabular}
\end{center}
\caption{Tabela com os resultados do observador P referentes ao Vídeo 4.}
\label{tab:video4P}
\end{table}

\begin{table}[H]
\begin{center}
\begin{tabular}{|c||c||c|}
\hline
\multicolumn{3}{|c|}{Observador M}  \\ \hline \hline
Tempo(s) & Acontecimento & Vagas Ocupadas\\ \hline
0 & ROI 1: Seções 15,16,20 e 21 ocupadas. & 2 \\ \hline
6 & ROI 1: Seções 18 e 19 ocupadas. & 3 \\ \hline
17 & ROI 1: Seções 21 e 22 liberadas & 2 \\ \hline
28 & ROI 2: Seções 12,13 e 14 ocupadas & 3 \\
\hline
\end{tabular}
\end{center}
\caption{Tabela com os resultados do observador M referentes ao Vídeo 4.}
\label{tab:video4M}
\end{table}

\begin{table}[H]
\begin{center}
\begin{tabular}{|c||c||c|}
\hline
\multicolumn{2}{|c|}{DVE}  \\ \hline \hline
Tempo(s) & Acontecimento & Vagas Ocupadas\\ \hline
0 & ROI 1: Seções 15,16,20 e 21 ocupadas. & 2 \\ \hline
6 & ROI 1: Seções 18 e 19 ocupadas. & 3 \\ \hline
11 & ROI 1: Seção 22 ocupada. & 3 \\ \hline
14 & ROI 1: Seção 22 liberada. & 3 \\ \hline
23 & ROI 1: Seção 20 liberada. & 2 \\ \hline
28 & ROI 2: Seções 12 e 13 ocupadas. & 3 \\
\hline
\end{tabular}
\end{center}
\caption{Tabela com os resultados do \textit{DVE} referentes ao Vídeo 4.}
\label{tab:video3}
\end{table}

\begin{table}[H]
\begin{center}
\begin{tabular}{|c||c||c|}
\hline
\multicolumn{3}{|c|}{Acertos - seções}  \\ \hline \hline
Observador & Acertos & Taxa de acertos \\ \hline
F & 1750 & 97,22\% \\  \hline
P & 1749 & 97,16\% \\ \hline
M & 1772 & 98,44\% \\ \hline
Média & 1757 & 97,61\% \\
\hline
\hline
\multicolumn{3}{|c|}{Acertos - vagas}  \\ \hline \hline
Observador & Acertos & Taxa de acertos \\ \hline
F & 22 & 73,33\% \\  \hline
P & 22 & 73,33\% \\ \hline
M & 24 & 80\% \\ \hline
Média & 22,66 & 75,55\% \\
\hline
\end{tabular}
\end{center}
\caption{Tabela com o número de acertos e a taxa de acertos do \textit{DVE} nos dois critérios avaliados no Vídeo 4.}
\label{tab:rvideo4}
\end{table}

Este vídeo reforça a dificuldade do programa de classificar corretamente as seções ocupadas pelo veículo cinza. Porém a dificuldade parece não acontecer quando este carro estaciona na região inferior, quando o programa determina que o veículo ocupa as mesmas seções que os observadores humanos.

Há um atraso na desocupação de uma vaga, indicada por volta dos $18$ segundos pelos observadores mas aos $23$ segundos pelo programa. O erro ocorre por causa da criação de uma vaga composta apenas pela seção $20$. Apesar dos cuidados que o modo complexo toma para evitar este acontecimento, quando uma nova vaga é demarcada pelas seções $21$, $22$ e $23$, que se sobrepõe parcialmente a vaga detectada no início do vídeo, composta pelas seções $20$ e $21$. Como neste caso a seção da intersecção passa a fazer parte da vaga nova, a seção $20$ passa a representar uma vaga. Como a liberação desta seção só ocorre aos $23$ segundos, o número de vagas ocupadas indicado é errado por este intervalo de tempo.


\subsection{Vídeo 5}

Um veículo desocupa a sua vaga sainda pela parte superior da tela. O vídeo possui uma duração de $10$ segundos ou $600$ possíveis acertos.

\begin{figure}[H]
\centering
\begin{subfigure}{.5\textwidth}
\centering
\includegraphics[width=.8\linewidth]{Video5Inicio}
\caption{}
\end{subfigure}\
\begin{subfigure}{.5\textwidth}
\centering
\includegraphics[width=.8\linewidth]{Video5Fim}
\caption{}
\end{subfigure}
\centering
\caption{(a) O momento inicial do vídeo 5; (b) O momento final do vídeo 5.}%
\label{}%
\end{figure}

\begin{table}[H]
\begin{center}
\begin{tabular}{|c||c||c|}
\hline
\multicolumn{3}{|c|}{Observador F}  \\ \hline \hline
Tempo(s) & Acontecimento & Vagas Ocupadas\\ \hline
0 & ROI 1: Seções 15,16,18 e 19 ocupadas. & 3 \\
 & ROI 2: Seções 12 e 13 ocupadas. &  \\ \hline
3 & ROI 1: Seções 15 e 16 liberadas. & 2 \\
\hline
\end{tabular}
\end{center}
\caption{Tabela com os resultados do observador F referentes ao Vídeo 5.}
\label{tab:video5F}
\end{table}

\begin{table}[H]
\begin{center}
\begin{tabular}{|c||c||c|}
\hline
\multicolumn{3}{|c|}{Observador P}  \\ \hline \hline
Tempo(s) & Acontecimento & Vagas Ocupadas\\ \hline
0 & ROI 1: Seções 15,16,18 e 19 ocupadas. & 3 \\
 & ROI 2: Seções 12,13 e 14 ocupadas. &  \\ \hline
4 & ROI 1: Seções 15 e 16 liberadas. & 2 \\
\hline
\end{tabular}
\end{center}
\caption{Tabela com os resultados do observador P referentes ao Vídeo 5.}
\label{tab:video5P}
\end{table}

\begin{table}[H]
\begin{center}
\begin{tabular}{|c||c||c|}
\hline
\multicolumn{3}{|c|}{Observador M}  \\ \hline \hline
Tempo(s) & Acontecimento & Vagas Ocupadas\\ \hline
0 & ROI 1: Seções 15,16,18 e 19 ocupadas. & 3 \\
 & ROI 2: Seções 12,13 e 14 ocupadas. &  \\ \hline
3 & ROI 1: Seções 15 e 16 liberadas. & 2 \\
\hline
\end{tabular}
\end{center}
\caption{Tabela com os resultados do observador M referentes ao Vídeo 5.}
\label{tab:video5M}
\end{table}

\begin{table}[H]
\begin{center}
\begin{tabular}{|c||c||c|}
\hline
\multicolumn{3}{|c|}{DVE}  \\ \hline \hline
Tempo(s) & Acontecimento & Vagas Ocupadas\\ \hline
0 & ROI 1: Seções 15,16,18 e 19 ocupadas. & 3 \\
 & ROI 2: Seções 12 e 13 ocupadas. &  \\ \hline
1 & ROI 1: Seção 17 ocupada. & 3 \\ \hline
3 & ROI 1: Seções 15 e 16 liberadas. & 2 \\ \hline
4 & ROI 1: Seção 17 liberada. & 2 \\
\hline
\end{tabular}
\end{center}
\caption{Tabela com os resultados do \textit{DVE} referentes ao Vídeo 5.}
\label{tab:video5}
\end{table}

\begin{table}[H]
\begin{center}
\begin{tabular}{|c||c||c|}
\hline
\multicolumn{3}{|c|}{Acertos - seções}  \\ \hline
Observador & Acertos & Taxa de acertos \\ \hline
F & 597 & 99,50\% \\  \hline
P & 585 & 97,50\% \\ \hline
M & 587 & 97,83\% \\ \hline
Média & 1757 & 97,61\% \\
\hline
\hline
\multicolumn{3}{|c|}{Acertos - vagas}  \\ \hline \hline
Observador & Acertos & Taxa de acertos \\ \hline
F & 10 & 100\% \\  \hline
P & 9 & 90\% \\ \hline
M & 10 & 100\% \\ \hline
Média & 9,66 & 96,66\% \\
\hline
\end{tabular}
\end{center}
\caption{Tabela com o número de acertos e a taxa de acertos do \textit{DVE} nos dois critérios avaliados no Vídeo 5.}
\label{tab:rvideo5}
\end{table}


Um caso de testes simples, onde os erros ocorreram principalmente por causa da natureza subjetiva do que configura uma seção ocupada. O \textit{DVE} classifica a seção $17$ como ocupada e os observadores não. Por outro lado, dois dos observadores acharam que a seção $14$ da ROI $2$ estava ocupada, descordando do programa. Contudo, essas discordâncias não afetam a quantidade de vagas ocupadas.

\subsection{Vídeo 6}

Neste vídeo o veículo amarelo entra na cena pela parte inferior da tela. O veículo branco desocupa sua vaga e sai da tela pelo lado esquerdo. O vídeo tem $13$ segundos totalizando $780$ acertos possíveis.

\begin{figure}[H]
\centering
\begin{subfigure}{.5\textwidth}
\centering
\includegraphics[width=.8\linewidth]{Video6Inicio}
\caption{}
\end{subfigure}\
\begin{subfigure}{.5\textwidth}
\centering
\includegraphics[width=.8\linewidth]{Video6Fim}
\caption{}
\end{subfigure}
\centering
\caption{(a) O momento inicial do vídeo 6; (b) O momento final do vídeo 6.}%
\label{}%
\end{figure}

\begin{table}[H]
\begin{center}
\begin{tabular}{|c||c||c|}
\hline
\multicolumn{3}{|c|}{Observador F}  \\ \hline \hline
Tempo(s) & Acontecimento & Vagas Ocupadas \\ \hline
0 & ROI 1: Seções 18 e 19 ocupadas. & 2 \\
 & ROI 2: Seções 12 e 13 ocupadas. &  \\ \hline
5 & ROI 2: Seções 14 e 15 ocupadas. & 3 \\ \hline
8 & ROI 1:Seçoes 18 e 19 liberadas. & 2 \\
\hline

\end{tabular}
\end{center}
\caption{Tabela com os resultados do observador F referentes ao Vídeo 6.}
\label{tab:video6F}
\end{table}

\begin{table}[H]
\begin{center}
\begin{tabular}{|c||c||c|}
\hline
\multicolumn{3}{|c|}{Observador P}  \\ \hline \hline
Tempo(s) & Acontecimento & Vagas Ocupadas \\ \hline
0 & ROI 1: Seções 18 e 19 ocupadas. & 2 \\
 & ROI 2: Seções 12 e 13 ocupadas. &  \\ \hline
5 & ROI 1: Seções 15,16 e 17 ocupadas. & 3 \\ \hline
9 & ROI 2: Seções 18 e 19 liberadas. & 2 \\
\hline

\end{tabular}
\end{center}
\caption{Tabela com os resultados do observador P referentes ao Vídeo 6.}
\label{tab:video6P}
\end{table}

\begin{table}[H]
\begin{center}
\begin{tabular}{|c||c||c|}
\hline
\multicolumn{3}{|c|}{Observador M}  \\ \hline \hline
Tempo(s) & Acontecimento & Vagas Ocupadas \\ \hline
0 & ROI 1: Seções 18 e 19 ocupadas. & 2 \\
 & ROI 2: Seções 12 e 13 ocupadas. &  \\ \hline
4 & ROI 1: Seções 15,16 e 17 ocupadas. & 3 \\ \hline
8 & ROI 2: Seções 18,19 liberadas. & 2\\
\hline

\end{tabular}
\end{center}
\caption{Tabela com os resultados do observador M referentes ao Vídeo 6.}
\label{tab:video6M}
\end{table}

\begin{table}[H]
\begin{center}
\begin{tabular}{|c||c||c|}
\hline
\multicolumn{3}{|c|}{DVE}  \\ \hline \hline
Tempo(s) & Acontecimento & Vagas Ocupadas \\ \hline
0 & ROI 1: Seções 18 e 19 ocupadas. & 2 \\
 & ROI 2: Seções 12 e 13 ocupadas. &  \\ \hline
5 & ROI 2: Seções 15 e 16 ocupadas. & 3 \\ \hline
7 & ROI 2: Seção 17 ocupada. & 3 \\ \hline
11 & ROI 1: Seções 18 e 19 liberadas. & 2 \\
\hline
\end{tabular}
\end{center}
\caption{Tabela com os resultados do \textit{DVE} referentes ao Vídeo 6.}
\label{tab:video6}
\end{table}

\begin{table}[H]
\begin{center}
\begin{tabular}{|c||c||c|}
\hline
\multicolumn{3}{|c|}{Acertos - seções}  \\ \hline
Observador & Acertos & Taxa de acertos \\ \hline
F & 752 & 96,40\% \\  \hline
P & 774 & 99,23\% \\ \hline
M & 767 & 98,33\% \\ \hline
Média & 764,33 & 97,99\% \\
\hline
\hline
\multicolumn{3}{|c|}{Acertos - vagas}  \\ \hline \hline
Observador & Acertos & Taxa de acertos \\ \hline
F & 10 & 76,92\% \\  \hline
P & 11 & 84,61\% \\ \hline
M & 9 & 69,23\% \\ \hline
Média & 10 & 76,92\% \\
\hline
\end{tabular}
\end{center}
\caption{Tabela com o número de acertos e a taxa de acertos do \textit{DVE} nos dois 
critérios avaliados no Vídeo 6.}
\label{tab:rvideo6}
\end{table}


Um caso de testes aonde a avaliação do \textit{DVE} se mantém estável e concorda quase plenamente com a avaliação dos humanos. Os atraso da detecção da liberação das seções $18$ e $19$ da ROI $2$ e da desocupação da vaga representada por estas seções se deve ao fato de que o movimento do veículo saindo só é considerado finalizado depois que o veículo sai do campo de visão da câmera, o que ocorre alguns momentos depois que os observadores humanos consideraram que o veículo saiu da vaga.

\subsection{Vídeo 7}

Neste vídeo um dos carros sai da cena pela esquerda. O vídeo tem $17$ segundos de duração e portanto $1020$ acertos possíveis.

\begin{figure}[H]
\centering
\begin{subfigure}{.5\textwidth}
\centering
\includegraphics[width=.8\linewidth]{Video7Inicio}
\caption{}
\end{subfigure}\
\begin{subfigure}{.5\textwidth}
\centering
\includegraphics[width=.8\linewidth]{Video7Fim}
\caption{}
\end{subfigure}
\centering
\caption{(a) O momento inicial do vídeo 7; (b) O momento final do vídeo 7.}%
\label{}%
\end{figure}

\begin{table}[H]
\begin{center}
\begin{tabular}{|c||c||c|}
\hline
\multicolumn{3}{|c|}{Observador F}  \\ \hline \hline
Tempo(s) & Acontecimento & Vagas Ocupadas \\ \hline
0 & ROI 2: Seções 13,14,16 e 17 ocupadas. & 2 \\ \hline
4 & ROI 2: Seções 13 e 14 liberadas. & 1 \\
\hline
\end{tabular}
\end{center}
\caption{Tabela com os resultados do observador F referentes ao Vídeo 7.}
\label{tab:video7F}
\end{table}

\begin{table}[H]
\begin{center}
\begin{tabular}{|c||c||c|}
\hline
\multicolumn{3}{|c|}{Observador P}  \\ \hline \hline
Tempo(s) & Acontecimento & Vagas Ocupadas \\ \hline
0 & ROI 2: Seções 12,13,14,16 e 17 ocupadas. & 2 \\ \hline
6 & ROI 2: Seções 12,13,14 liberadas. & 1 \\ 
\hline
\end{tabular}
\end{center}
\caption{Tabela com os resultados do observador P referentes ao Vídeo 7.}
\label{tab:video7P}
\end{table}

\begin{table}[H]
\begin{center}
\begin{tabular}{|c||c||c|}
\hline
\multicolumn{3}{|c|}{Observador M}  \\ \hline \hline
Tempo(s) & Acontecimento & Vagas Ocupadas \\ \hline
0 & ROI 2: Seções 12,13,14,16 e 17 ocupadas. & 2 \\ \hline
5 & ROI 2: Seções 12,13,14 liberadas. & 1 \\ 
\hline
\end{tabular}
\end{center}
\caption{Tabela com os resultados do observador M referentes ao Vídeo 7.}
\label{tab:video7M}
\end{table}

\begin{table}[H]
\begin{center}
\begin{tabular}{|c||c||c|}
\hline
\multicolumn{3}{|c|}{DVE}  \\ \hline \hline
Tempo(s) & Acontecimento & Vagas Ocupadas \\ \hline
0 & ROI 2: Seções 13,14,16 e 17 ocupadas. & 2 \\ \hline
1 & ROI 2: Seção 12 ocupada. & 2 \\ \hline
2 & ROI 2: Seção 14 liberada. & 2 \\ \hline
5 & ROI 2: Seções 12 e 13 liberadas. & 1 \\
\hline
\end{tabular}
\end{center}
\caption{Tabela com os resultados do \textit{DVE} referentes ao Vídeo 7.}
\label{tab:video7}
\end{table}

\begin{table}[H]
\begin{center}
\begin{tabular}{|c||c||c|}
\hline
\multicolumn{3}{|c|}{Acertos - seções}  \\ \hline
Observador & Acertos & Taxa de acertos \\ \hline
F & 1013 & 99,31\% \\  \hline
P & 1014 & 99,41\% \\ \hline
M & 1016 & 99,60\% \\ \hline
Média & 1014,33 & 99,44\% \\
\hline
\hline
\multicolumn{3}{|c|}{Acertos - vagas}  \\ \hline \hline
Observador & Acertos & Taxa de acertos \\ \hline
F & 16 & 94,11\% \\  \hline
P & 16 & 94,11\% \\ \hline
M & 17 & 100\% \\ \hline
Média & 16,33 & 96,07\% \\
\hline
\end{tabular}
\end{center}
\caption{Tabela com o número de acertos e a taxa de acertos do \textit{DVE} nos dois 
critérios avaliados no Vídeo 7.}
\label{tab:rvideo7}
\end{table}

O programa concordou com os observadores nesse vídeo, exibindo apenas um leve atraso para a classificação correta da seção $12$ e uma liberação levemente precipitada da seção $14$. A ocupação das vagas é determinada de forma quase idêntica a dos observadores.

\subsection{Vídeo 8}

No oitavo e último caso de testes, um carro branco passa pela região central da imagem sem estacionar em nenhuma vaga e depois um carro cinza estaciona na região inferior. O vídeo tem $23$ segundos de duração e $1380$ acertos possíveis.

\begin{figure}[H]
\centering
\begin{subfigure}{.5\textwidth}
\centering
\includegraphics[width=.8\linewidth]{Video8Inicio}
\caption{}
\end{subfigure}\
\begin{subfigure}{.5\textwidth}
\centering
\includegraphics[width=.8\linewidth]{Video8Fim}
\caption{}
\end{subfigure}
\centering
\caption{(a) O momento inicial do vídeo 8; (b) O momento final do vídeo 8.}%
\label{}%
\end{figure}

\begin{table}[H]
\begin{center}
\begin{tabular}{|c||c||c|}
\hline
\multicolumn{3}{|c|}{Observador F}  \\ \hline \hline
Tempo(s) & Acontecimento & Vagas Ocupadas\\ \hline
0 & ROI 2: Seções 16 e 17 ocupadas. & 1 \\ \hline
20 & ROI 2: Seções 22 e 23 ocupadas. & 2 \\
\hline
\end{tabular}
\end{center}
\caption{Tabela com os resultados do observador F referentes ao Vídeo 8.}
\label{tab:video8F}
\end{table}


\begin{table}[H]
\begin{center}
\begin{tabular}{|c||c||c|}
\hline
\multicolumn{3}{|c|}{Observador P}  \\ \hline \hline
Tempo(s) & Acontecimento & Vagas Ocupadas\\ \hline
0 & ROI 2: Seções 16 e 17 ocupadas. & 1 \\ \hline
20 & ROI 2: Seções 22 , 23 e 24 ocupadas. & 2 \\
\hline
\end{tabular}
\end{center}
\caption{Tabela com os resultados do observador P referentes ao Vídeo 8.}
\label{tab:video8P}
\end{table}


\begin{table}[H]
\begin{center}
\begin{tabular}{|c||c||c|}
\hline
\multicolumn{3}{|c|}{Observador M}  \\ \hline \hline
Tempo(s) & Acontecimento & Vagas Ocupadas\\ \hline
0 & ROI 2: Seções 16 e 17 ocupadas. & 1 \\ \hline
20 & ROI 2: Seções 22,23 e 24 ocupadas. & 2 \\
\hline
\end{tabular}
\end{center}
\caption{Tabela com os resultados do observador M referentes ao Vídeo 8.}
\label{tab:video8M}
\end{table}


\begin{table}[H]
\begin{center}
\begin{tabular}{|c||c||c|}
\hline
\multicolumn{3}{|c|}{DVE}  \\ \hline \hline
Tempo(s) & Acontecimento & Vagas Ocupadas\\ \hline
0 & ROI 2: Seções 16 e 17 ocupadas. & 1 \\ \hline
20 & ROI 2: Seções 21,22,23 e 24 ocupadas. & 2 \\ \hline
21 & ROI 2: Seção 21 liberada. & 2 \\
\hline
\end{tabular}
\end{center}
\caption{Tabela com os resultados do \textit{DVE} referentes ao Vídeo 8.}
\label{tab:video8}
\end{table}

\begin{table}[H]
\begin{center}
\begin{tabular}{|c||c||c|}
\hline
\multicolumn{3}{|c|}{Acertos - seções}  \\ \hline\hline
Observador & Acertos & Taxa de acertos \\ \hline
F & 1376 & 99,71\% \\  \hline
P & 1379 & 99,92\% \\ \hline
M & 1379 & 99,92\% \\ \hline
Média & 1378 & 99,85\% \\
\hline
\hline
\multicolumn{3}{|c|}{Acertos - vagas}  \\ \hline \hline
Observador & Acertos & Taxa de acertos \\ \hline
F & 23 & 100\% \\  \hline
P & 23 & 100\% \\ \hline
M & 23 & 100\% \\ \hline
Média & 23 & 100\% \\
\hline
\end{tabular}
\end{center}
\caption{Tabela com o número de acertos e a taxa de acertos do \textit{DVE} nos dois 
critérios avaliados no Vídeo 8.}
\label{tab:rvideo8}
\end{table}

O programa avalia as seções de forma quase idêntica aos humanos. Aos $20s$, a classificação que ocorre regularmente determina que quatro seções estão sendo ocupadas pelo veículo. Um segundo depois porém, quando o movimento do veículo termina, o erro é corrigido e o programa volta a acertar completamente.












	\chapter{Conclusão}\label{cap:conclusao}

A tarefa de se encontrar vagas em estacionamentos tem se tornado cada vez mais difícil com o aumento da frota de veículos das grandes cidades. Essa busca por um espaço livre para estacionar o carro pode durar muito tempo, criando gastos que se acumulam até valores altíssimos.

Buscando minimizar estes problemas, diversos sistemas de mapeamento de estacionamento e detecção de vagas livres foram desenvolvidos. Algumas soluções são implementadas nos próprios veículos \cite{schmid2011parking}, mas as mais interessantes para uma solução geral do problema são aquelas que detectam a quantidade de vagas livres em uma região do estacionamento e disponibilizam essa informação para todos os seus usuários. Em garagens, sensores de vários tipos podem ser utilizados \cite{kianpisheh2012smart}\cite{wolff2006parking}\cite{lee2008intelligent}, mas estas soluções não são muito adequadas para aplicação em estacionamentos descobertos. 

Para estes casos, uma solução que se mostrou adequada e de fácil implementação foi o uso de câmeras de vídeo e algoritmos de processamento de imagens e visão computacional. Este trabalho apresentou um algoritmo para ser utilizado em um sistema como estes, chamado de \textit{Detector de vagas em estacionamentos abertos}, abreviado como \textit{DVE}. O algoritmo recebe as imagens de câmera de vídeo coloridas montadas em postes de luz e usa uma rede neural artificial combinada com uma técnica de rastreamento dos veículos através do fluxo óptico para mapear e determinar a ocupação das vagas que aparecem na imagem.

O \textit{DVE} é especialmente adequado para estacionamentos descobertos com postes de luz em intervalos de distância regular e com poucas ou nenhuma obstrução visual. Quando testado em vídeos que simulavam a captura de uma câmera instalada em um destes postes, o programa apresentou resultados bastante semelhantes a resultados determinados por um observador humano. Ele apresentou dificuldade em classificar corretamente regiões ocupadas por carros com coloração semelhante a do asfalto quando iluminados de certa maneira. Além disso apresentou a possibilidade de detecção errônea de vagas, o que fazia que fossem acusadas uma ocupação maior do que a real nos vídeos.

Apesar de alguns erros, o \textit{DVE} é resistente a pequenos erros de classificação da rede neural na maioria dos casos e até aos movimentos do equipamento de captura, já que na maioria das vezes continuou detectando o número correto de vagas ocupadas na imagem apesar destes empecilhos.

Por fim, o programa se mostrou promissor e capaz de ajudar de forma simples usuários de estacionamento a encontrarem vagas com mais facilidade. Ainda é necessário, porém, fazer um refinamento do sistema de mapeamento do \textit{DVE}, de forma que ele seja capaz de criar um mapa preciso do estacionamento monitorado e disponibilizar mais informações aos usuários, como tempo médio de espera ou a posição exata das vagas livres. Um outro possíel trabalho futuro é a adaptação do \textit{DVE} para que ele funcione com imagens de câmera em ângulos olíquos ao estacionamento, permitindo uma instalação ainda mais fácil e em estacionamentos mais variados. Já foram feitos experimentos neste sentido usando seções quadradas menores ao invés das seções verticais e a mesma rede neural com resultados premilinares promissores, como exemplificado na figura \ref{fig:preliminares}

\begin{figure}%
\centering
\includegraphics[width=8cm]{preliminar}%
\caption{Resultados preliminares de um possível trabalho futuro.}%
\label{fig:preliminares}%
\centering
\end{figure}










  
  % ...

  \postextual
  \bibliographystyle{plain}
  \bibliography{bibliografia}

\end{document}

